\documentclass{article}
\usepackage{noweb}
\usepackage[pdftex]{graphicx}
%\usepackage{times}
\addtolength{\textwidth}{1in}
\addtolength{\oddsidemargin}{-.5in}
\setlength{\evensidemargin}{\oddsidemargin}

\newcommand{\myfig}[1]{\resizebox{\textwidth}{!}
                        {\includegraphics{figure/#1.pdf}}}

\noweboptions{breakcode}
\title{The \emph{pedigree} functions in R}
\author{Terry Therneau and Elizabeth Atkinson}

\begin{document}
\maketitle
\section{Introduction}
The pedigree routines came out of a simple need -- to quickly draw a
pedigree structure on the screen, within R, that was ``good enough'' to
help with debugging the actual routines of interest, which were those for
fitting mixed effecs Cox models to large family data.  As such the routine
had compactness and automation as primary goals; complete annotation
(monozygous twins, multiple types of affected status) and most certainly
elegance were not on the list.  Other software could do that much
better.

It therefore came as a major surprise when these routines proved useful
to others.  Through their constant feedback, application to more
complex pedigrees, and ongoing requests for one more feature, the routine has 
become what it is today.  This routine is still not 
suitable for really large pedigrees, nor for heavily inbred ones such as in
animal studies, and will likely not evolve in that way.  The authors' fondest%'
hope is that others will pick up the project.

\section{Pedigree}
The pedigree function is the first step, creating an object of class
\emph{pedigree}.  
It accepts the following input
\begin{description}
  \item[id] A numeric or character vector of subject identifiers.
  \item[dadid] The identifier of the father.
  \item[momid] The identifier of the mother.
  \item[sex] The gender of the individual.  This can be a numeric variable
    with codes of 1=male, 2=female, 3=unknown, 4=terminated, or NA=unknown.
    A character or factor variable can also be supplied containing
    the above; the string may be truncated and of arbitrary case.  A sex
    value of 0=male 1=female is also accepted.
  \item[status] Optional, a numeric variable with 0 = censored and 1 = dead.
  \item[relationship] Optional, a matrix or data frame with three columns.
    The first two contain the identifier values of the subject pairs, and
    the third the code for their relationship:
    1 = Monozygotic twin, 2= Dizygotic twin, 3= Twin of unknown zygosity,
    4 = Spouse.  
  \item[famid] Optional, a numeric or character vector of family identifiers.
\end{description}

The {\tt{}famid} variable is placed last as it was a later addition to the
code; thus prior invocations of the function that use positional 
arguments won't be affected.
If present, this allows a set of pedigrees to be generated, one per
family.  The resultant structure will be an object of class
{\tt{}pedigreeList}.

Note that a factor variable is not listed as one of the choices for the
subject identifier. This is on purpose.  Factors
were designed to accomodate character strings whose values came from a limited
class -- things like race or gender, and are not appropriate for a subject
identifier.  All of their special properties as compared to a character
variable turn out to be backwards for this case, in particular a memory
of the original level set when subscripting is done.
However, due to the awful decision early on in S to automatically turn every
character into a factor --- unless you stood at the door with a club to
head the package off --- most users have become ingrained to the idea of
using them for every character variable. 
(I encourage you to set the global option stringsAsFactors=FALSE to turn
off autoconversion -- it will measurably improve your R experience).
Therefore, to avoid unnecessary hassle for our users 
the code will accept a factor as input for the id variables, but
the final structure does not retain it.  
Gender and relation do become factors.  Status follows the pattern of the 
survival routines and remains an integer.

We will describe the code in a set of blocks.
\nwfilename{all.nw}\nwbegincode{1}\moddef{pedigree}\endmoddef
pedigree <- function(id, dadid, momid, sex, affected, status, relations,
                     famid) \{
    \LA{}pedigree-error\RA{}
    \LA{}pedigree-parent\RA{}
    \LA{}pedigree-create\RA{}
    \LA{}pedigree-extra\RA{}
    if (missing(famid)) class(temp) <- 'pedigree'
    else class(temp) <- 'pedigreeList'
    temp
    \}
\LA{}pedigree-subscript\RA{}
\nwendcode{}\nwbegindocs{2}\nwdocspar

\subsection{Data checks}
The code starts out with some checks on the input data.  
Is it all the same length, are the codes legal, etc.
\nwenddocs{}\nwbegincode{3}\moddef{pedigree-error}\endmoddef
n <- length(id)
if (length(momid) != n) stop("Mismatched lengths, id and momid")
if (length(dadid) != n) stop("Mismatched lengths, id and momid")
if (length(sex  ) != n) stop("Mismatched lengths, id and sex")

# Don't allow missing id values
if (any(is.na(id))) stop("Missing value for the id variable")
if (!is.numeric(id)) \{
    id <- as.character(id)
    if (length(grep('^ *$', id)) > 0)
    stop("A blank or empty string is not allowed as the id variable")
  \}

# Allow for character/numeric/factor in the sex variable
if(is.factor(sex))
        sex <- as.character(sex)
codes <- c("male","female", "unknown", "terminated")
if(is.character(sex)) sex<- charmatch(casefold(sex, upper = FALSE), codes, 
                                  nomatch = 3)  

# assume either 0/1/2/4 =  female/male/unknown/term, or 1/2/3/4
#  if only 1/2 assume no unknowns
if(min(sex) == 0)
        sex <- sex + 1
sex <- ifelse(sex < 1 | sex > 4, 3, sex)
if(all(sex > 2))
        stop("Invalid values for 'sex'")
    else if(mean(sex == 3) > 0.25)
        warning("More than 25% of the gender values are 'unknown'")
sex <- factor(sex, 1:4, labels = codes)
\nwendcode{}\nwbegindocs{4}\nwdocspar

Create the variables descibing a missing father and/or mother,
which is what we expect both for people at the top of the
pedigree and for marry-ins, \emph{before} adding in the family
id information.  
It's easier to do it first.
If there are multiple families in the pedigree, make a working set of
identifiers that are of the form `family/subject'.
Family identifiers can be factor, character, or numeric.
\nwenddocs{}\nwbegincode{5}\moddef{pedigree-error}\plusendmoddef
if (is.numeric(id)) \{
    nofather <- (is.na(dadid) | dadid==0)
    nomother <- (is.na(momid) | momid==0)
    \}
else \{
    nofather <- (is.na(dadid) | dadid=="")
    nomother <- (is.na(momid) | momid=="")
    \}

if (!missing(famid)) \{
    if (any(is.na(famid))) stop("The family id cannot contain missing values")
    if (is.factor(famid) || is.character(famid)) \{
        if (length(grep('^ *$', famid)) > 0)
            stop("The family id cannot be a blank or empty string")
        \}
    #Make a temporary new id from the family and subject pair
    oldid <-id
    id <- paste(as.character(famid), as.character(id), sep='/')
    dadid <- paste(as.character(famid), as.character(dadid), sep='/')
    momid <- paste(as.character(famid), as.character(momid), sep='/')
    \}

if (any(duplicated(id))) \{
    duplist <- id[duplicated(id)]
    msg.n <- min(length(duplist), 6)
    stop(paste("Duplicate subject id:", duplist[1:msg.n]))
    \}
\nwendcode{}\nwbegindocs{6}\nwdocspar

Next check that any mother or father identifiers are found in the identifier
list, and are of the right sex.
Subjects who don't have a mother or father are founders.  For those people %'
both of the parents should be missing.

\nwenddocs{}\nwbegincode{7}\moddef{pedigree-parent}\endmoddef
findex <- match(dadid, id, nomatch = 0)
if(any(sex[findex] != "male")) \{
    who <- unique((id[findex])[sex[findex] != "male"])
    msg.n <- 1:min(5, length(who))  #Don't list a zillion
    stop(paste("Id not male, but is a father:", 
               paste(who[msg.n], collapse= " ")))
    \}

if (any(findex==0 & !nofather)) \{
    who <- dadid[which(findex==0 & !nofather)]
    msg.n <- 1:min(5, length(who))  #Don't list a zillion
    stop(paste("Vale of 'dadid' not found in the id list", 
               paste(who[msg.n], collapse= " ")))
    \}
    
mindex <- match(momid, id, nomatch = 0)
if(any(sex[mindex] != "female")) \{
    who <- unique((id[mindex])[sex[mindex] != "female"])
    msg.n <- 1:min(5, length(who))
    stop(paste("Id not female, but is a mother:", 
               paste(who[msg.n], collapse = " ")))
    \}

if (any(mindex==0 & !nomother)) \{
    who <- momid[which(mindex==0 & !nomother)]
    msg.n <- 1:min(5, length(who))  #Don't list a zillion
    stop(paste("Value of 'momid' not found in the id list", 
               paste(who[msg.n], collapse= " ")))
    \}

if (any(mindex==0 & findex!=0) || any(mindex!=0 & findex==0))
    stop("Subjects must have both a father and mother, or have neither")

if (!missing(famid)) \{
    if (any(famid[mindex] != famid[mindex>0])) \{
        who <- (id[mindex>0])[famid[mindex] != famid[mindex>0]]
        msg.n <- 1:min(5, length(who))
        stop(paste("Mother's family != subject's family", 
                   paste(who[msg.n], collapse=" ")))
        \}
    if (any(famid[findex] != famid[findex>0])) \{
        who <- (id[findex>0])[famid[findex] != famid[findex>0]]
        msg.n <- 1:min(5, length(who))
        stop(paste("Father's family != subject's family", 
                   paste(who[msg.n], collapse=" ")))
        \}
    \}
\nwendcode{}\nwbegindocs{8}\nwdocspar

\subsection{Creation}
Now, paste the parts together into a basic pedigree.
The fields for father and mother are not the identifiers of
the parents, but their row number in the structure.
\nwenddocs{}\nwbegincode{9}\moddef{pedigree-create}\endmoddef
if (missing(famid))
    temp <- list(id = id, findex=findex, mindex=mindex, sex = sex)
else temp<- list(famid=famid, id=oldid, findex=findex, mindex=mindex, 
                 sex=sex)
\nwendcode{}\nwbegindocs{10}\nwdocspar

The last part is to check out the optional features,
affected status, survival status, and relationships.

Update by Jason Sinnwell, 5/2011: Allow missing values (NA) in the 
affected status matrix. 

\nwenddocs{}\nwbegincode{11}\moddef{pedigree-extra}\endmoddef
if (!missing(affected)) \{
    if (is.matrix(affected))\{
        if (nrow(affected) != n) stop("Wrong number of rows in affected")
        if (is.logical(affected)) affected <- 1* affected
        \} 
    else \{
        if (length(affected) != n)
            stop("Wrong length for affected")

        if (is.logical(affected)) affected <- as.numeric(affected)
        if (is.factor(affected))  affected <- as.numeric(affected) -1
        \}
    if(max(affected, na.rm=TRUE) > min(affected, na.rm=TRUE)) 
      affected <- affected - min(affected, na.rm=TRUE)
    if (!all(affected==0 | affected==1 | is.na(affected)))
                stop("Invalid code for affected status")
    temp$affected <- affected
    \}

if(!missing(status)) \{
    if(length(status) != n)
        stop("Wrong length for affected")
    if (is.logical(status)) status <- as.integer(status)
    if(any(status != 0 & status != 1))
        stop("Invalid status code")
    temp$status <- status
    \}

if (!missing(relations)) \{
    if (!missing(famid)) \{
        if (is.matrix(relations)) \{
            if (ncol(relations) != 4) 
                stop("Relations matrix must have 3 columns + famid")
            id1 <- relations[,1]
            id2 <- relations[,2]
            code <- relations[,3]
            famid <- relations[,4]
            \}
        else if (is.data.frame(relations)) \{
            id1 <- relations$id1
            id2 <- relations$id2
            code <- relations$code
            famid <- relations$famid
            if (is.null(id1) || is.null(id2) || is.null(code) ||is.null(famid)) 
            stop("Relations data must have id1, id2, family id and code")
            \}
        else stop("Relations argument must be a matrix or a dataframe")
        \}
    else \{
        if (is.matrix(relations)) \{
            if (ncol(relations) != 3) 
                stop("Relations matrix must have 3 columns")
            id1 <- relations[,1]
            id2 <- relations[,2]
            code <- relations[,3]
            \}
        else if (is.data.frame(relations)) \{
            id1 <- relations$id1
            id2 <- relations$id2
            code <- relations$code
            if (is.null(id1) || is.null(id2) || is.null(code)) 
                stop("Relations data frame must have id1, id2, and code")
            \}
        else stop("Relations argument must be a matrix or a list")
        \}
    
    if (!is.numeric(code))
        code <- match(code, c("MZ twin", "DZ twin", "UZ twin", "spouse"))
    else code <- factor(code, levels=1:4,
                        labels=c("MZ twin", "DZ twin", "UZ twin", "spouse"))
    if (any(is.na(code)))
        stop("Invalid relationship code")
     
    # Is everyone in this relationship in the pedigree?
    if (!missing(famid)) \{
        temp1 <- match(paste(as.character(famid), as.character(id1), sep='/'), 
                       id, nomatch=0)
        temp2 <- match(paste(as.character(famid), as.character(id2), sep='/'),
                       id, nomatch=0)
        \}
    else \{
        temp1 <- match(id1, id, nomatch=0)
        temp2 <- match(id2, id, nomatch=0)
        \}
    
    if (any(temp1==0 | temp2==0))
        stop("Subjects in relationships that are not in the pedigree")
    if (any(temp1==temp2)) \{
        who <- temp1[temp1==temp2]
        stop(paste("Subject", id[who], "is their own spouse or twin"))
        \}

    # Check, are the twins really twins?
    ncode <- as.numeric(code)
    if (any(ncode<3)) \{
        twins <- (ncode<3)
        if (any(momid[temp1[twins]] != momid[temp2[twins]]))
            stop("Twins found with different mothers")
        if (any(dadid[temp1[twins]] != dadid[temp2[twins]]))
            stop("Twins found with different fathers")
        \}
    # Check, are the monozygote twins the same gender?
    if (any(code=="MZ twin")) \{
        mztwins <- (code=="MZ twin")
        if (any(sex[temp1[mztwins]] != sex[temp2[mztwins]]))
            stop("MZ Twins with different genders")
        \}

    ##Use id index as id1 and id2
    if (!missing(famid)) \{
        temp$relation <- data.frame(famid=famid, id1=temp1, id2=temp2, code=code)
         
        \}
    else temp$relation <- data.frame(id1=temp1, id2=temp2, code=code)
    \}
\nwendcode{}\nwbegindocs{12}\nwdocspar

The final structure will be in the order of the original data, and all the
components except {\tt{}relations} will have the
same number of rows as the original data.


\subsection{Subcripting}

Subscripting of a pedigree list extracts one or more families from the
list.  We treat character subscripts in the same way that dimnames on
a matrix are used.  
The 'trick' is to create an integer vector with the appropriate names,
and subscript it.  Then we inherit all of the default R rules without
having to think them through.
The only exception is when family id is a numeric: when the user
says [4] do they mean the fourth family in the list or family '4'?
The user is responsible to say ['4'] in this case.

In a normal vector invalid subscripts give an NA, e.g. (1:3)[6], but
since there is no such object as an ``NA pedigree'', we emit an error
for this.
The {\tt{}drop} argument has no meaning for pedigrees, but must to be
a defined argument of any subscript method; we simply ignore it.
For both methods updating the father/mother is a minor nuisance;
since they must are integer indices to rows they must be
recreated after selection.  
\nwenddocs{}\nwbegincode{13}\moddef{pedigree-subscript}\endmoddef
"[.pedigreeList" <- function(x, ..., drop=F) \{
    if (length(list(...)) != 1) stop ("Only 1 subscript allowed")
    ufam <- unique(x$famid)
    nfamily <- length(ufam)
    temp <- 1:nfamily
    names(temp) <- ufam
    indx <- temp[..1]   #which families to keep

    if (any(is.na(indx))) 
            stop(paste("Familiy", (..1[is.na(indx)])[1], "not found"))

    keep <- which(x$famid %in% names(indx))  #which rows to keep
    for (i in c('id', 'famid', 'sex'))
        x[[i]] <- (x[[i]])[keep]
    
    kept.rows <- (1:length(x$findex))[keep]
    x$findex <- match(x$findex[keep], kept.rows, nomatch=0)
    x$mindex <- match(x$mindex[keep], kept.rows, nomatch=0)
    
    #optional components
    if (!is.null(x$status)) x$status <- x$status[keep]
    if (!is.null(x$affected)) \{
        if (is.matrix(x$affected)) x$affected <- x$affected[keep,,drop=FALSE]
        else x$affected <- x$affected[keep]
        \}
    if (!is.null(x$relation)) \{
        keep <- !is.na(match(x$relation$famid, names(indx)))
        if (any(keep)) \{
            x$relation <- x$relation[keep,]
            ##Update twin id indexes
            x$relation$id1 <- match(x$relation$id1, kept.rows, nomatch=0)
            x$relation$id2 <- match(x$relation$id2, kept.rows, nomatch=0)
            ##If only one family chosen, remove famid
            if (length(indx)==1) \{x$relation$famid <- NULL\}
            \}
        \}
    if (length(indx)==1)  class(x) <- 'pedigree'  #only one family chosen
    else class(x) <- 'pedigreeList'
    x
    \}
\nwendcode{}\nwbegindocs{14}\nwdocspar

For a pedigree, the subscript operator extracts a subset of individuals.
We disallow selections that retain only 1 of a subject's parents, since    %'
they cause plotting trouble later.
Relations are worth keeping only if both parties in the relation were
selected.

\nwenddocs{}\nwbegincode{15}\moddef{pedigree-subscript}\plusendmoddef
"[.pedigree" <- function(x, ..., drop=F) \{
    if (length(list(...)) > 1)  stop ("Only 1 subscript allowed")
    n <- length(x$id)
    temp <- 1:n
    names(temp) <- x$id   #ids are known to be unique, and not factors
    i <- temp[..1]   #integer index

    z <- list(id=x$id[i],findex=match(x$findex[i], i, nomatch=0),
              mindex=match(x$mindex[i], i, nomatch=0),
             sex=x$sex[i])
    if (!is.null(x$affected)) z$affected <- x$affected[i,, drop=F]
    if (!is.null(x$famid)) z$famid <- x$famid[keep]

    if (!is.null(x$relation)) \{
        indx1 <- match(x$relation$id1, i, nomatch=0)
        indx2 <- match(x$relation$id2, i, nomatch=0)
        keep <- (indx1 >0 & indx2 >0)  #keep only if both id's are kept
        if (any(keep))
          z$relation <- x$relation[keep,]
        z$relation$id1 <- indx1
        z$relation$id2 <- indx2
        \}
    
    if (!is.null(x$hints)) \{
        temp <- list(order= x$hints$order[i])
        if (!is.null(x$hints$spouse)) \{
            indx1 <- match(x$hints$spouse[,1], i, nomatch=0)
            indx2 <- match(x$hints$spouse[,2], i, nomatch=0)
            keep <- (indx1 >0 & indx2 >0)  #keep only if both id's are kept
            if (any(keep))
                temp$spouse <- cbind(indx1[keep], indx2[keep],
                                     x$hints$spouse[keep,3])
            \}
        z$hints <- temp
        \}
q    
    if (any(z$findex==0 & z$mindex>0) | any(z$findex>0 & z$mindex==0))
        stop("A subpedigree cannot choose only one parent of a subject")
    class(z) <- 'pedigree'
    z
    \}
\nwendcode{}\nwbegindocs{16}\nwdocspar

\subsection{Printing}
It usually doesn't make sense to print a pedigree, since the id is just   %'
a repeat of the input data and the family connections are pointers.
Thus we create a simple summary.

\nwenddocs{}\nwbegincode{17}\moddef{print.pedigree}\endmoddef
print.pedigree <- function(x, ...) \{
    cat("Pedigree object with", length(x$id), "subjects")
    if (!is.null(x$famid)) cat(", family id=", x$famid[1], "\\n")
    else cat("\\n")
    \}

print.pedigreeList <- function(x, ...) \{
    cat("Pedigree list with", length(x$id), "total subjects in",
        length(unique(x$famid)), "families\\n")
    \}
\nwendcode{}\nwbegindocs{18}\nwdocspar
\section{Kinship matrices}
The kinship matrix is foundational for random effects models with family
data.  
For $n$ subjects it is an $n \times n$ matrix whose $ij$ element contains
the expected fraction of alleles that are identical by descent (IBD)for subject
$i$ and $j$.
Note that the diagonal elements of the matrix will be 0.5 not 1: if I randomly
sample two alleles of one of your genes, with replacement, 1/2 the time I get
a father/father or mother/mother pair (IBD) and the other 1/2 the time get
one of each.  
The truely astute reader will recognize that values >.5 can occur due to
inbreeding, but I'll leave that discussion for others.                    %'

The algorithm used is that found in K Lange, 
\emph{Mathematical and Statistical  Methods for Genetic Analysis}, 
Springer 1997, page 71--72.
It starts by setting the rows/columns for founders to .5 time the identity
matrix, they require no further processing.  
Parents must be processed before their children, and then a child's        %'
kinship is a sum of the kinship's for his/her parents.                     %'

Start by using the {\tt{}kindepth} routine to label each subject's depth in   %'
the pedigree.  
The initial matrix suffices for all those of depth 0, then process
depth 1, etc.
This guarrantees that parent's precede children.                            %'
Founders are given a fake parent with id of n+1 who is unrelated to 
himself -- a little trick that avoids some if-else logic.

The most non-obvious part of the algorithm is the inner loop over {\tt{}i}.
It looks like a natural candidate for S-vectorization, but you cannot.
The key is {\tt{}kmat[mom,]\ +\ kmat[dad,]\ }: as we walk through a set of
siblings these vectors change, the $i$th element goes from 0 to the
appropriate value for that sib.  The dependence of each sib on prior
ones is what creates the correct between-sib correlation terms. 
The impact of the inner loop is not so dreadful, however, since this
function is run once per family.  A study may have thousands of
subjects but individual families within it are more modest in size.

The program can be called with a pedigree, a pedigree list, or
raw data.  The first argument is {\tt{}id} instead of the more generic {\tt{}x}
for backwards compatability.
\nwenddocs{}\nwbegincode{19}\moddef{kinship}\endmoddef
kinship <- function(id, ...) \{
    UseMethod('kinship')
    \}

kinship.default <- function(id, dadid, momid) \{
    n <- length(id)
    if (any(duplicated(id))) stop("All id values must be unique")
    kmat <- diag(n+1) /2
    kmat[n+1,n+1]    <- 0 

    pdepth <- kindepth(id, dadid, momid)
    mrow <- match(momid, id, nomatch=n+1) #row number of the mother
    drow <- match(dadid, id, nomatch=n+1) #row number of the dad 

    for (depth in 1:max(pdepth)) \{
        indx <- (1:n)[pdepth==depth]
        for (i in indx) \{
            mom <- mrow[i]
            dad <- drow[i]
            kmat[i,]  <- kmat[,i] <- (kmat[mom,] + kmat[dad,])/2
            kmat[i,i] <- (1+ kmat[mom,dad])/2
            \}
        \}
    
    kmat <- kmat[1:n,1:n]
    dimnames(kmat) <- list(id, id)
    kmat
    \}
\nwendcode{}\nwbegindocs{20}\nwdocspar

The method for a pedigree object is not quite trivial.  Since the
mother and father are already indexed into the id list it has 
two lines that are different, those that create mrow and drow.
Otherwise it is a complete repeat.
\nwenddocs{}\nwbegincode{21}\moddef{kinship}\plusendmoddef
kinship.pedigree <- function(id, ...) \{
    n <- length(id$id)
    if (any(duplicated(id$id))) stop("All id values must be unique")
    kmat <- diag(n+1) /2
    kmat[n+1,n+1]    <- 0 

    pdepth <- kindepth(id)
    mrow <- ifelse(id$mindex ==0, n+1, id$mindex)
    drow <- ifelse(id$findex ==0, n+1, id$findex)

    for (depth in 1:max(pdepth)) \{
        indx <- (1:n)[pdepth==depth]
        for (i in indx) \{
            mom <- mrow[i]
            dad <- drow[i]
            kmat[i,]  <- kmat[,i] <- (kmat[mom,] + kmat[dad,])/2
            kmat[i,i] <- (1+ kmat[mom,dad])/2
            \}
        \}
    
    kmat <- kmat[1:n,1:n]
    dimnames(kmat) <- list(id$id, id$id)
    kmat
    \}    
\nwendcode{}\nwbegindocs{22}\nwdocspar
For the Minnesota Family Cancer Study there are 461 families and 26050
subjects.  The raw kinship matrix would be 26050 by 26050 which is over
5 terabytes of memory, something that clearly won't work within S.       %'
The solution is to store the overall matrix as a bdsmatrix object (block
diagonal symmetric).  
Each family forms a single block.  For this study we have
{\tt{}n\ <-\ table(minnbreast{\char36}famid);\ sum(n*(n+1)/2)} or 1.07 million entries;
only the lower half of each matrix is stored.

The code is actually even more frugal: subjects in the pedigree who have
no genetic connections, i.e., those who have married in but have no children
in the data set, can be treated as individual $1 \times 1$ blocks, and are
placed at the tail of each family.  We can't mix them in the middle and
retain the simple block structure.  
Why don't we use {\tt{}(i\ in\ famlist)} below?  A numeric subscript of {\tt{}[9]} 
selects the ninth family, not the family labeled as 9, so a numeric
family id would not act as we wished.
If all of the subject ids are unique, across all families, the final
matrix is labeled with the subject id, otherwise it is labeled with
family/subject.
\nwenddocs{}\nwbegincode{23}\moddef{kinship}\plusendmoddef
kinship.pedigreeList <- function(id, ...) \{
    plist <- id  #rename, to make the code below easier to read
    if (any(duplicated(plist$id))) addfamid <- TRUE
    else                           addfamid <- FALSE
    famlist <- unique(plist$famid)
    blocksize <- NULL
    blocks <- NULL
    newid <- NULL
    for (i in 1:length(famlist)) \{
        tped <- plist[i]  #pedigree for this family
        kmat <- kinship(tped)
        related <- (rowSums(kmat>0) >1)  #this person is related to someone else
        if (any(related)) \{
            blocksize <- c(blocksize, sum(related))
            blocks <- c(blocks, kmat[related, related])
            kmat <- kmat[!related, !related]
            \}
        if (any(!related)) \{
            blocksize <- c(blocksize, rep(1, sum(!related)))
            blocks <- c(blocks, diag(kmat))
            \}
        if (addfamid) \{
            temp <- paste(famlist[i], c(tped$id[related], tped$id[!related]), 
                          sep='/')
            newid <- c(newid, temp)
            \}
        else newid <- c(newid, tped$id[related], tped$id[!related])
        \}
    bdsmatrix(blocksize, blocks, dimnames=list(newid, newid))
    \}                           
\nwendcode{}\nwbegindocs{24}\nwdocspar

The older {\tt{}makekinship} function,
from before the creation of pedigreeList objects,
accepts the raw identifier data, along with a special family code
for unrelated subjects, as produced by the {\tt{}makefamid} function.
All the unrelated subjects are put at the front of the kinship matrix
in this case rather than at the end of a family.
Because unrelateds get put into a fake family, we cannot create a
rational family/subject identifier; the id must be unique across
families.
Like most routines, this starts out with a collection of error checks.
\nwenddocs{}\nwbegincode{25}\moddef{makekinship}\endmoddef
makekinship <- function(famid, id, father.id, mother.id, unrelated=0) \{
    n <- length(famid)
    if (length(id)    != n) stop("Mismatched lengths: famid and id")
    if (length(mother.id) != n) stop("Mismatched lengths: famid and mother.id")
    if (length(father.id) != n) stop("Mismatched lengths: famid and father.id")
    if (any(is.na(famid)))  stop("One or more subjects with missing family id")
    if (any(is.na(id)))     stop("One or more subjects with a missing id")
    if (is.numeric(famid)) \{
        if (any(famid <0))      stop("Invalid family id, must be >0")
        \}

    if (any(duplicated(id))) stop("Subject ids must be unique")

    famlist <- sort(unique(famid))  #same order as the counts table
    idlist <- id            # will be overwritten, but this makes it the
                            #  correct data type and length
    counts <- table(famid)
    cumcount <- cumsum(counts)    
     if (any(famid==unrelated)) \{
        # Assume that those with famid of 0 are unrelated uniques
        #   (usually the marry-ins)
        temp <- match(unrelated, names(counts))
        nzero <- counts[temp]    
        counts <- counts[-temp]
        famlist <- famlist[famlist != unrelated]
        idlist[1:nzero] <- id[famid== unrelated]
        cumcount <- cumsum(counts) + nzero
        \}
    else nzero <- 0
    
    blockn <- counts*(counts+1)/2   #size of storage for each block
    n2 <- sum(blockn)       # total amount needed
    bdata <- double(n2)
    j <- cumsum(blockn)     
    for (i in 1:length(counts)) \{
        who <- (famid == famlist[i])
        if (sum(who) ==1) bdata[j[i]] <- 0.5  # family of size 1
        else \{
            temp <- kinship(id[who], mother.id[who], father.id[who])
            bdata[seq(to=j[i], length=blockn[i])] <- temp[row(temp)>=col(temp)]
            \}
        idlist[seq(to=cumcount[i], length=counts[i])] <- id[who]
        \}

    bdsmatrix(blocksize=c(rep(1,nzero), counts),
              blocks =  c(rep(.5,nzero), bdata),
              dimnames=list(idlist, idlist))
    \}
\nwendcode{}\nwbegindocs{26}\nwdocspar




\section{Pedigree alignment}
An \emph{aligned} pedigree is an object that contains a pedigree along
with a set of information that allows for pretty plotting.
This information consists of two parts: 
a set of vertical and horizontal plotting coordinates along with the
identifier of the subject to be plotted at each position,
and a list of connections to be made between parent/child, spouse/spouse,
and twin/twin.
Creating this aligment turned out to be one of the more difficult parts
of the project, and is the area where significant further work could be
done.  
All the routines in this section completely ignore the {\tt{}id} component
of a pedigree; everyone is indexed solely by their row number in the object.

\subsection{Hints}

The first part of the work has to do with a {\tt{}hints} list for each
pedigree.  It consists of 3 parts:
\begin{itemize}
  \item The left to right order in which founders should be processed.
  \item The order in which siblings should be listed within a family.
  \item For selected spouse pairs, who is on the left/right, and which of the
    two should be the anchor, i.e., determine where the marriage is plotted.
    \end{itemize}
The default starting values for all of these are simple: founders are 
processed in the order in which they appear in the data set, 
children appear in the order they are found in the data set,
husbands are to the left of their wives, and a marriage is plotted
at the leftmost spouse.
A simple example where we want to bend these rules is when two families
marry, and the pedigrees for both extend above the wedded pair.  
In the joint pedigree the
pair should appear as the right-most child in the left hand family, and
as the left-most child in the right hand family.
With respect to founders, assume that a family has three lineages with
a marriage between 1 and 2, and another between 2 and 3.  In the joint
pedigree the sets should be 1, 2, 3 from left to right.  

The hints consist of a list with two components.
The first is a vector of numbers of the same length as the pedigree,
used to order the female founders and to order siblings within
family.  For subjects not part of either of these the value can be 
arbitrary.  
The second is a 3 column matrix of spouse pairs, each row indicates the
left-hand member of the pair, the right-hand member, and which of the two
is the anchor, i.e., directly connected to thier parent.
Double and triple marriages can start to get interesting.


The {\tt{}autohint} routine is used to create an initial hints list.
It is a part of the general intention to make the routine do
``pretty good'' drawings automatically.                 
The basic algorithm is trial and error. 
\begin{itemize}
  \item Start with the simplest possible hints (user input is accepted)
  \item Call align.pedigree to see how this works out
  \item Fix any spouses that are not next to each other but could be.
  \item Any fix on the top level mixes up everything below, so we do the
    fixes one level at a time.
\end{itemize}
The routine makes no attempt to reorder founders.  It just isn't smart enough%'
to figure that out.

The first thing to be done is to check on twins.  They are a nuisance, since
twins need to move together.  The {\tt{}ped{\char36}relation} object has a factor in it, 
so first turn that into numeric.
\nwenddocs{}\nwbegincode{27}\moddef{autohint}\endmoddef
autohint <- function(ped, hints) \{
    if (!is.null(ped$hints)) return(ped$hints)  #nothing to do
    n <- length(ped$id)
    depth <- kindepth(ped, align=TRUE)

    if (is.null(ped$relation)) relation <- NULL
    else  relation <- cbind(as.matrix(ped$relation[,1:2]), 
                            as.numeric(ped$relation[,3]))
    if (!is.null(relation) && any(relation[,3] <4)) \{
        temp <- (relation[,3] < 4)
        twinlist <- unique(c(relation[temp,1:2]))  #list of twin id's 
        twinrel  <- relation[temp,,drop=F]
        
        twinset <- rep(0,n)
        twinord <- rep(1,n)
        for (i in 2:length(twinlist)) \{
            # Now, for any pair of twins on a line of twinrel, give both
            #  of them the minimum of the two ids
            # For a set of triplets, it might take two iterations for the
            #  smallest of the 3 numbers to "march" across the threesome.
            #  For quads, up to 3 iterations, for quints, up to 4, ....
            newid <- pmin(twinrel[,1], twinrel[,2])
            twinset[twinrel[,1]] <- newid
            twinset[twinrel[,2]] <- newid
            twinord[twinrel[,2]] <- pmax(twinord[twinrel[,2]], 
                                         twinord[twinrel[,1]]+1)
            \}   
        \}
    else \{
        twinset <- rep(0,n)
        twinrel <- NULL
        \}
    \LA{}autohint-shift\RA{}
    \LA{}autohint-init\RA{}
    \LA{}autohint-fixup\RA{}
    list(order=horder, spouse=sptemp)    
    \}
\nwendcode{}\nwbegindocs{28}\nwdocspar

Next is an internal function that  rearranges someone to be
the leftmost or rightmost of his/her siblings.  The only
real complication is twins -- if one of them moves the other has to move too.  
And we need to keep the monozygotics together within a band of triplets.
Algorithm: if the person to be moved is part of a twinset, 
first move all the twins to the left end (or right
as the case may be), then move all the monozygotes to the
left, then move the subject himself to the left.
\nwenddocs{}\nwbegincode{29}\moddef{autohint-shift}\endmoddef
shift <- function(id, sibs, goleft, hint, twinrel, twinset) \{
    if (twinset[id]> 0)  \{ 
        shift.amt <- 1 + diff(range(hint[sibs]))  # enough to avoid overlap
        twins <- sibs[twinset[sibs]==twinset[id]]
        if (goleft) 
         hint[twins] <- hint[twins] - shift.amt
        else hint[twins] <- hint[twins] + shift.amt
            
        mono  <- any(twinrel[c(match(id, twinrel[,1], nomatch=0),
                           match(id, twinrel[,2], nomatch=0)),3]==1)
        if (mono) \{
        #
        # ok, we have to worry about keeping the monozygotics
        #  together within the set of twins.
        # first, decide who they are, by finding those monozygotic
            #  with me, then those monozygotic with the results of that
            #  iteration, then ....  If I were the leftmost, this could
            #  take (#twins -1) iterations to get us all
            #
        monoset <- id
        rel2 <- twinrel[twinrel[,3]==1, 1:2, drop=F]
        for (i in 2:length(twins)) \{
            newid1 <- rel2[match(monoset, rel2[,1], nomatch=0),2]
            newid2 <- rel2[match(monoset, rel2[,2], nomatch=0),1]
            monoset <- unique(c(monoset, newid1, newid2))
            \}
        if (goleft) 
               hint[monoset]<- hint[monoset] - shift.amt
        else   hint[monoset]<- hint[monoset] + shift.amt
        \}
        \}

    #finally, move the subject himself
    if (goleft) hint[id] <- min(hint[sibs]) -1   
    else        hint[id] <- max(hint[sibs]) +1

    hint[sibs] <- rank(hint[sibs])  # aesthetics -- no negative hints
    hint
    \}
\nwendcode{}\nwbegindocs{30}\nwdocspar

Now, get an ordering of the pedigree to "look at".  
The numbers start at 1 on each level.
We don't need the final ``prettify" step, hence align=F.
If there is a hints structure entered, we retain it's non-zero entries,
otherwise people are put into the order of the data set. 
We allow the hints input to be only an order vector
Twins are
then further reordered.
\nwenddocs{}\nwbegincode{31}\moddef{autohint-init}\endmoddef
if (!missing(hints)) \{
    if (is.vector(hints)) hints <- list(order=hints)
    if (is.matrix(hints)) hints <- list(spouse=hints)
    if (is.null(hints$order)) horder <- integer(n)
    else horder <- hints$order
    \}
else horder <- integer(n)

for (i in unique(depth)) \{
    who <- (depth==i & horder==0)
    if (any(who)) horder[who] <- 1:sum(who)
    \}

if (any(twinset>0)) \{
    # First, make any set of twins a cluster: 6.01, 6.02, ...
    #  By using fractions, I don't have to worry about other sib's values
    for (i in unique(twinset)) \{
        if (i==0) next
        who <- (twinset==i)
        horder[who] <- mean(horder[who]) + twinord[who]/100
        \}

    # Then reset to integers
    for (i in unique(ped$depth)) \{
        who <- (ped$depth==i)
        horder[who] <- rank(horder[who])  #there should be no ties
        \}
    \}

if (!missing(hints)) sptemp <- hints$spouse
else sptemp <- NULL
plist <- align.pedigree(ped, packed=TRUE, align=FALSE, 
                        hints=list(order=horder, spouse=sptemp))
\nwendcode{}\nwbegindocs{32}\nwdocspar
The result coming back from align.pedigree is a set of vectors and
matrices:
\begin{description}
  \item[n] vector, number of entries per level
  \item[nid] matrix, one row per level, numeric id of the subject plotted
    here
  \item[spouse] integer matrix, one row per level, subject directly to my
    right is my spouse (1), a double marriage (2), or neither (0).
  \item[fam] matrix, link upward to my parents, or 0 if no link.
\end{description}

\begin{figure}
  \myfig{autohint1}
  \caption{A simple pedigree before (left) and after (right) the
    autohint computations.}
  \label{fig:auto1}
\end{figure}

Now, walk down through the levels one by one.
A candidate subject is one who appears twice on the level, once
under his/her parents and once somewhere else as a spouse.
Move this person and spouse the the ends of their sibships and
add a marriage hint.
Figure \ref{fig:auto1} shows a simple case.  The input data set has
the subjects ordered from 1--11, the left panel is the result without
hints which processes subjects in the order encountered.
The return values from {\tt{}align.pedigree} have subject 9 shown twice.
The first is when he is recognized as the spouse of subject 4, the second
as the child of 6--7.

The basic logic is
\begin{enumerate}
  \item Find a subject listed multiple times on a line (assume it is a male).
    This means that he has multiple connections, usually one to his parents and
    the other to a spouse tied to her parents.  (If the
    spouse were a marry-in she would have been placed alongside and there
    would be no duplication.)
  \item Say subject x is listed at locations 2, 8, and 12.  We look at one
    pairing at a time, either 2-8 or 8-12.  Consider the first one.
    \begin{itemize}
      \item If position 2 is associated with siblings, rearrange them to
        put subject 2 on the right.  If it is associated with a spouse at
        this location, put that spouse on the right of her siblings.
      \item Repeat the work for position 8, but moving targets to the left.
      \item At either position, if it is associated with a spouse then
        add a marriage.  If both ends of the marriage are anchored, i.e.,
        connected to a family, then either end may be listed as the anchor
        in the output; follow the suggestion of the duporder routine.  If
        only one is, it is usually better to anchor it there, so that the
        marriage is processed by{\tt{}align.pedigree} when that family is.
        (At least I think so.)
    \end{itemize}
\end{enumerate}
This logic works 9 times out of 10, at least for human pedigrees.
We'll look at more complex cases below when looking at the {\tt{}duporder}   %'
(order the duplicates)
function, which returns a matrix with columns 1 and 2 being a pair
of duplicates, and 3 a direction.
Note that in the following code {\tt{}idlist} refers to the row numbers of
each subject in the pedigree, not to their label {\tt{}ped{\char36}id}.
\nwenddocs{}\nwbegincode{33}\moddef{autohint-fixup}\endmoddef
\LA{}autohint-find\RA{}
\LA{}autohint-duporder\RA{}
maxlev <- nrow(plist$nid)
for (lev in 1:maxlev) \{
    idlist <- plist$nid[lev,1:plist$n[lev]] #subjects on this level
    dpairs <- duporder(idlist, plist, lev, ped)  #duplicates to be dealt with
    if (nrow(dpairs)==0) next;  
    for (i in 1:nrow(dpairs)) \{
        anchor <- spouse <- rep(0,2)
        for (j in 1:2) \{
            direction <- c(FALSE, TRUE)[j]
            mypos <- dpairs[i,j]
            if (plist$fam[lev, mypos] >0) \{
                # Am connected to parents at this location
                anchor[j] <- 1  #familial anchor
                sibs <- idlist[findsibs(mypos, plist, lev)]
                if (length(sibs) >1) 
                    horder <- shift(idlist[mypos], sibs, direction, 
                                    horder, twinrel, twinset)
                \}
            else \{
                #spouse at this location connected to parents ?
                spouse[j] <- findspouse(mypos, plist, lev, ped)
                if (plist$fam[lev,spouse[j]] >0) \{ # Yes they are
                    anchor[j] <- 2  #spousal anchor
                    sibs <- idlist[findsibs(spouse[j], plist, lev)]
                    if (length(sibs) > 1) 
                        horder <- shift(idlist[spouse[j]], sibs, direction, 
                                    horder, twinrel, twinset)
                    \}
                \}
            \}

        # add the marriage(s)
        id1 <- idlist[dpairs[i,1]]  # i,1 and i,2 point to the same person
        if (anchor[1] !=1) \{
            id2 <- idlist[spouse[1]]
            if (anchor[2] ==1)      temp <- c(id2, id1, dpairs[i,3])
            else if (anchor[1] ==0) temp <- c(id2, id1, 2)
            else                    temp <- c(id2, id1, 1)
            sptemp <- rbind(sptemp, temp)
            \}
        if (anchor[2] !=1) \{
            id2 <- idlist[spouse[2]]
            if  (anchor[1] ==1)     temp <- c(id1, id2, dpairs[i,3])
            else if (anchor[2] ==0) temp <- c(id1, id2, 1)             
            else                    temp <- c(id1, id2, 2)
            sptemp <- rbind(sptemp, temp)
            \}
      \}
    #
    # Recompute, since this shifts things on levels below
    #
    plist <- align.pedigree(ped, packed=TRUE, align=FALSE, 
                            hints=list(order=horder, spouse=sptemp))   
    \}
\nwendcode{}\nwbegindocs{34}\nwdocspar

For the case shown in figure \ref{fig:align1} the {\tt{}duporder} function
will return a single row array with values (2, 6, 1), the first two
being the positions of the duplicated subject.  
The anchor will be 2 since that is the copy connected to parents
The direction is TRUE, since the spouse is to the left of the anchor point.
The id is 9, sibs are 8, 9, 10, and the shift function will create position
hints of 2,1,3, which will cause them to be listed in the order 9, 8, 10.

The value of spouse is 3 (third position in the row), subjects 3,4, and 5
are reordered, and finally the line (4,9,1) is added to the sptemp 
matrix.  
In this particular case the final element could be a 1 or a 2, since both
are connected to their parents.

\begin{figure}
  \myfig{autohint2}
  \caption{Two test pedigrees showing the initial attempt on the left,
    and that after autohint on the right.}
  \label{fig:align2}
\end{figure}

Figure \ref{fig:align2} shows a more complex case with several arcs.
In the upper left is a double marry-in.
The {\tt{}anchor} variable in the above code
will be zero: neither of the two copies of subject 11 is anchored to
a parent.
The left and right sets of sibs are reordered (even though the left
one does not need it), and two lines are added to the sptemp matrix:
(5,11,1) and (11,9,2).

On the upper right is a pair of overlapping arcs.
In the final tree we want to put sibling 28 to the right of 29 since
that will allow one node to join, but if we process the subjects in
lexical order the code will first shift 28 to the right and then later
shift over 29.
The duporder function tries to order the duplicates into a matrix
so that the closest ones are processed last.  The definition of close
is based first on whether the families touch, and second on the
actual distance.
The third column of the matrix hints at whether the marriage should
be plotted at the left (1) or right (2) position of the pair.  The
goal for this is to spread apart families of cousins; in the
example to not have the children of 28/31 plotted under the 21/22
grandparents, and those for 29/32 under the 25/26 grandparents. 
The logic for this column is very ad hoc: put children near the edges.
\nwenddocs{}\nwbegincode{35}\moddef{autohint-duporder}\endmoddef
duporder <- function(idlist, plist, lev, ped) \{
    temp <- table(idlist)
    if (all(temp==1)) return (matrix(0L, nrow=0, ncol=3))
    
    # make an intial list of all pairs's positions
    # if someone appears 4 times they get 3 rows
    npair <- sum(temp-1)
    dmat <- matrix(0L, nrow=npair, ncol=3)
    dmat[,3] <- 2; dmat[1:(npair/2),3] <- 1
    i <- 0
    for (id in unique(idlist[duplicated(idlist)])) \{
        j <- which(idlist==id)
        for (k in 2:length(j)) \{
            i <- i+1
            dmat[i,1:2] <- j[k + -1:0]
            \}
        \}
    if (nrow(dmat)==1) return(dmat)  #no need to sort it
    
    # families touch?
    famtouch <- logical(npair)
    for (i in 1:npair) \{
        if (plist$fam[lev,dmat[i,1]] >0) 
             sib1 <- max(findsibs(dmat[i,1], plist, lev))
        else \{
            spouse <- findspouse(dmat[i,1], plist, lev, ped)
            ##If spouse is marry-in then move on without looking for sibs
                if (plist$fam[lev,spouse]==0) \{famtouch[i] <- F; next\}
            sib1 <- max(findsibs(spouse, plist, lev)) 
            \}
        
        if (plist$fam[lev, dmat[i,2]] >0)
            sib2 <- min(findsibs(dmat[i,2], plist, lev))
        else \{
            spouse <- findspouse(dmat[i,2], plist, lev, ped)
            ##If spouse is marry-in then move on without looking for sibs
                if (plist$fam[lev,spouse]==0) \{famtouch[i] <- F; next\}
            sib2 <- min(findsibs(spouse, plist, lev))
            \}
        famtouch[i] <- (sib2-sib1 ==1)
        \}
    dmat[order(famtouch, dmat[,1]- dmat[,2]),, drop=FALSE ]
    \}
\nwendcode{}\nwbegindocs{36}\nwdocspar

Finally, here are two helper routines.
Finding my spouse can be interesting -- suppose we have a listing with
Shirley, Fred, Carl, me on the line with the first three marked as
spouse=TRUE -- it means that she has been married to all 3 of us.
First we find the string from rpos to lpos that is a marriage block;
99\% of the time this will be of length 2 of course.  Then find
the person in that block who is opposite sex, and check that they
are connected.
The routine is called with a left-right position in the alignment
arrays and returns a position.
\nwenddocs{}\nwbegincode{37}\moddef{autohint-find}\endmoddef
findspouse <- function(mypos, plist, lev, ped) \{
    lpos <- mypos
    while (lpos >1 && plist$spouse[lev, lpos-1]) lpos <- lpos-1
    rpos <- mypos
    while(plist$spouse[lev, rpos]) rpos <- rpos +1
    if (rpos==lpos) stop("autohint bug 3")
    
    opposite <-ped$sex[plist$nid[lev,lpos:rpos]] != ped$sex[plist$nid[lev,mypos]]
    if (sum(opposite) != 1) stop("autohint bug 4")
    spouse <- (lpos:rpos)[opposite]
    spouse
    \}
\nwendcode{}\nwbegindocs{38}\nwdocspar

The findsibs function starts with a position and returns a positin as well.
\nwenddocs{}\nwbegincode{39}\moddef{autohint-find}\plusendmoddef
findsibs <- function(mypos, plist, lev) \{
    family <- plist$fam[lev, mypos]
    if (family==0) stop("autohint bug 6")
    which(plist$fam[lev,] == family)
    \}
\nwendcode{}\nwbegindocs{40}\nwdocspar


\subsection{Main routine}
\label{sect:alignped}
The main routine accepts has 5 arguments
\begin{description}
    \item[ped] a pedigree or pedigreeList object. In the case of
      the latter we loop over each family separately.
    \item[packed] do we allow branches of the tree to overlap?  
      If FALSE the drawing is much easier, but final drawing can
      take up a huge amount of space.  
    \item[width] the minimum width for a packed pedigree. This
      affects only small pedigrees, since the minimum possible
      width for a pedigree is the largest number of individiuals in
      one of the generations.
    \item[align] should the final step of alignment be done?  This
      tries to center children under parents, to the degree possible.
    \item a hints object.  This is normally blank and autohint
      is invoked. 
\end{description}
The result coming back from align.pedigree is a set of vectors and
matrices:
\begin{description}
  \item[n] vector, number of entries per level
  \item[nid] matrix, one row per level, numeric id of the subject plotted
    here
  \item[pos] the horizontal position for plotting
  \item[spouse] integer matrix, one row per level, subject directly to my
    right is my spouse (1), a double marriage (2), or neither (0).
  \item[fam] matrix, link upward to my parents, or 0 if no link.
\end{description}
\nwenddocs{}\nwbegincode{41}\moddef{align.pedigree}\endmoddef
align.pedigree <- function(ped, packed=TRUE, width=10,
                           align=TRUE, hints=ped$hints) \{
    if (class(ped)== 'pedigreeList') \{
        nped <- length(unique(ped$famid))
        alignment <- vector('list', nped)
        for (i in 1:nped) \{
            temp <- align.pedigree(ped[i], packed, width, align)
            alignment[[i]] <- temp$alignment
            \}
        ped$alignment <- alignment
        class(ped) <- 'pedigreeListAligned'
        return(ped)
        \}
    
    if (is.null(hints)) hints <- autohint(ped)
    else     hints <- check.hint(hints, ped$sex)
    
    \LA{}align-setup\RA{}
    \LA{}align-founders\RA{}
    \LA{}align-finish\RA{}
    \}
\nwendcode{}\nwbegindocs{42}\nwdocspar


Start with some setup.  
Throughout this routine the row number is used as a subject
id (ignoring the actual id label).
\begin{itemize}
  \item Check that everyone has either two
    parents or none (a singleton will just confuse us).
  \item Verify that the hints are correct.
  \item The relation data frame, if present, has a factor in it.  Turn
    that into numeric.
\item Create the {\tt{}spouselist} array.  This has 4 columns
  \begin{enumerate}
    \item Husband index (4= 4th person in the pedigree structure)
    \item Wife index
    \item Plot order: 1= husband left, 2=wife left
    \item Anchor: 1=left member, 2=right member, 0= not yet determined
      \end{enumerate}
  As the routine proceeds a spousal pair can be encountered
  multiple times; we take them out of this list when the ``connected''
  member is added to the pedigree so that no marriage gets added
  twice.  
\item To detect duplicates on the spouselist we need to create a
  unique (but temporary) spouse-pair id using a simple hash.
\end{itemize}

When importing data from autohint, it's spouse matrix has column 1 =    %'
subject plotted on the left, 2= subject plotted on the right.
The {\tt{}spouselist} array has 1=husband, 2=wife.  
Hence the clumsy looking ifelse below.  The autohint format is more
congenial to users, who might modify the output, the spouselist format
easier for the code.

\nwenddocs{}\nwbegincode{43}\moddef{align-setup}\endmoddef
n <- length(ped$id)
dad <- ped$findex; mom <- ped$mindex  #save typing
if (any(dad==0 & mom>0) || any(dad>0 & mom==0))
        stop("Everyone must have 0 parents or 2 parents, not just one")
level <- 1 + kindepth(ped, align=TRUE)

horder <- hints$order   # relative order of siblings within a family

if (is.null(ped$relation)) relation <- NULL
else  relation <- cbind(as.matrix(ped$relation[,1:2]), 
                        as.numeric(ped$relation[,3]))

if (!is.null(hints$spouse)) \{ # start with the hints list
    tsex <- ped$sex[hints$spouse[,1]]  #sex of the left member
    spouselist <- cbind(0,0,  1+ (tsex!='male'), 
                        hints$spouse[,3])
    spouselist[,1] <- ifelse(tsex=='male', hints$spouse[,1], hints$spouse[,2])
    spouselist[,2] <- ifelse(tsex=='male', hints$spouse[,2], hints$spouse[,1])
    \}
else spouselist <- matrix(0L, nrow=0, ncol=4)

if (!is.null(relation) && any(relation[,3]==4)) \{
    # Add spouses from the relationship matrix
    trel <- relation[relation[,3]==4,,drop=F]
    tsex <- ped$sex[trel[,1]]
    trel[tsex!='male',1:2] <- trel[tsex!='male',2:1]
    spouselist <- rbind(spouselist, cbind(trel[,1],
                                          trel[,2],
                                          0,0))
    \}
if (any(dad>0 & mom>0) ) \{
    # add parents
    who <- which(dad>0 & mom>0)
    spouselist <- rbind(spouselist, cbind(dad[who], mom[who], 0, 0))
    \}

hash <- spouselist[,1]*n + spouselist[,2]
spouselist <- spouselist[!duplicated(hash),, drop=F]
\nwendcode{}\nwbegindocs{44}\nwdocspar

The {\tt{}alignped} routine does the alignment using 3 co-routines:
\begin{description}
  \item[alignped1] called with a single subject, returns the subtree
    founded on this subject, as though it were the only tree
  \item[alignped2] called with a set of sibs, calls alignped1 and 
    alignped3 multiple times to create a joint pedigree
  \item[alignped3] given two side by side plotting structures, merge them
    into a single one
\end{description}
 
Call {\tt{}alignped1} sequentially with each founder pair and merge the
results.  
A founder pair is a married pair, neither of which has a father.

\nwenddocs{}\nwbegincode{45}\moddef{align-founders}\endmoddef
noparents <- (dad[spouselist[,1]]==0 & dad[spouselist[,2]]==0)
 ##Take duplicated mothers and fathers, then founder mothers
dupmom <- spouselist[noparents,2][duplicated(spouselist[noparents,2])] #Founding mothers with multiple marriages
dupdad <- spouselist[noparents,1][duplicated(spouselist[noparents,1])] #Founding fathers with multiple marriages
foundmom <- spouselist[noparents&!(spouselist[,1] %in% c(dupmom,dupdad)),2] # founding mothers
founders <-  unique(c(dupmom, dupdad, foundmom))    
founders <-  founders[order(horder[founders])]  #use the hints to order them
rval <- alignped1(founders[1], dad, mom, level, horder, 
                      packed=packed, spouselist=spouselist)

if (length(founders)>1) \{
    spouselist <- rval$spouselist
    for (i in 2:length(founders)) \{
        rval2 <- alignped1(founders[i], dad, mom,
                           level, horder, packed, spouselist)
        spouselist <- rval2$spouselist
        rval <- alignped3(rval, rval2, packed)
        \}
    \}
\nwendcode{}\nwbegindocs{46}\nwdocspar

Now finish up.  
There are 4 tasks to doS
\begin{enumerate}
  \item For convenience the lower level routines kept the spouse
    and nid arrays as a single object -- unpack them
  \item In the spouse array a 1 in position i indicates that subject
    i and i+1 are joined as a marriage.  If these two have a common
    ancestor change this to a 2, which indicates that a double line
    should be used in the plot.
  \item Add twins data to the output.
  \item Do final alignment
\end{enumerate}

\nwenddocs{}\nwbegincode{47}\moddef{align-finish}\endmoddef
#
# Unhash out the spouse and nid arrays
#
nid    <- floor(rval$nid)
spouse <- 1*(rval$nid != nid)
maxdepth <- nrow(nid)

# For each spouse pair, find out if it should be connected with
#  a double line.  This is the case if they have a common ancestor
ancestor <- function(me, momid, dadid) \{
    alist <- me
    repeat \{
        newlist <- c(alist, momid[alist], dadid[alist])
        newlist <- sort(unique(newlist[newlist>0]))
        if (length(newlist)==length(alist)) break
        alist <- newlist
        \}
    alist[alist!=me]
    \}
for (i in (1:length(spouse))[spouse>0]) \{
    a1 <- ancestor(nid[i], mom, dad)
    a2 <- ancestor(nid[i+maxdepth],mom, dad)  #matrices are in column order
    if (any(duplicated(c(a1, a2)))) spouse[i] <- 2
    \}
\nwendcode{}\nwbegindocs{48}\nwdocspar

The twins array is of the same shape as the spouse and nid arrays:
one row per level giving data for the subjects plotted on that row.
In this case they are
\begin{itemize}
  \item 0= nothing
  \item 1= the sib to my right is a monzygotic twin, 
  \item 2= the sib to my right is a dizygote,
  \item 3= the sib to my right is a twin, unknown zyogosity.
\end{itemize}
\nwenddocs{}\nwbegincode{49}\moddef{align-finish}\plusendmoddef
if (!is.null(relation) && any(relation[,3] < 4)) \{
    twins <- 0* nid
    who  <- (relation[,3] <4)
    ltwin <- relation[who,1]
    rtwin <- relation[who,2]
    ttype <- relation[who,3]
    
    # find where each of them is plotted (any twin only appears
    #   once with a family id, i.e., under their parents)
    ntemp <- ifelse(rval$fam>0, nid,0) # matix of connected-to-parent ids
    ltemp <- (1:length(ntemp))[match(ltwin, ntemp, nomatch=0)]
    rtemp <- (1:length(ntemp))[match(rtwin, ntemp, nomatch=0)]
    twins[pmin(ltemp, rtemp)] <- ttype
    \}
else twins <- NULL
\nwendcode{}\nwbegindocs{50}\nwdocspar
 
At this point the pedigree has been arranged, with the positions
in each row going from 1 to (number of subjects in the row).
(For a packed pedigree, which is the usual case).
Having everything pushed to the left margin isn't very
pretty, now we fix that.
Note that alignped4 wants a T/F spouse matrix: it doesn't care
  about your degree of relationship to the spouse.
\nwenddocs{}\nwbegincode{51}\moddef{align-finish}\plusendmoddef
if ((is.numeric(align) || align) && max(level) >1) 
    pos <- alignped4(rval, spouse>0, level, width, align)
else pos <- rval$pos

if (is.null(twins))
     list(n=rval$n, nid=nid, pos=pos, fam=rval$fam, spouse=spouse)
else list(n=rval$n, nid=nid, pos=pos, fam=rval$fam, spouse=spouse, 
              twins=twins)
\nwendcode{}\nwbegindocs{52}\nwdocspar
\subsection{alignped1}
This is the first of the three co-routines.
It is called with a single subject, and returns the subtree founded
on said subject, as though it were the only tree.  
We only go down the pedigree, not up.
Input arguments are
\begin{description}
  \item[nid] the numeric id of the subject in question
  \item[dad] points to the row of the father, 0=no father in pedigree
  \item[mom] points to the row of the mother
  \item[level] the plotting depth of each subject
  \item[horder] orders the kids within a sibship
  \item[packed] if true, everything is slid to the left
  \item[spouselist] a matrix of spouses
    \begin{itemize}
      \item col 1= pedigree index of the husband
      \item col 2= pedigree index of the wife
      \item col 3= 1:plot husband to the left, 2= wife to the left
      \item col 4= 1:left member is rooted here, 2=right member, 0=either
    \end{itemize}
\end{description}

The return argument is a set of matrices as described in 
section \ref{sect:alignped}, along with the spouselist matrix.
The latter has marriages removed as they are processed..

In this routine the {\tt{}nid} array consists of the final nid array + 1/2 of the
final spouse array.
The basic algorithm is simple.  
\begin{enumerate}
  \item Find all of the spouses for which {\tt{}x} is the anchor subject.  If
    there are none then return the trivial tree consisting of {\tt{}x} alone.
  \item For each marriage in the set, call {\tt{}alignped2} on the children
    and add this to the result.
\end{enumerate}
Note that the {\tt{}spouselist} matrix will only contain spouse pairs that
are not yet processed.
The logic for anchoring is slightly tricky.  First, if row 4 of
the spouselist matrix is 0, we anchor at the first opportunity, i.e. now..
Also note that if spouselist[,3]==spouselist[,4] it is
the husband who is the anchor (just write out the possibilities).

\nwenddocs{}\nwbegincode{53}\moddef{alignped1}\endmoddef
alignped1 <- function(x, dad, mom, level, horder, packed, spouselist)\{
    # Set a few constants
    maxlev <- max(level)
    lev <- level[x]
    n <- integer(maxlev)

    if (length(spouselist)==0)  spouse <- NULL
    else \{
        if (any(spouselist[,1]==x))\{
            sex <- 1                              # I'm male
            sprows <- (spouselist[,1]==x & (spouselist[,4] ==spouselist[,3] |
                                            spouselist[,4] ==0))
            spouse <- spouselist[sprows, 2] #ids of the spouses
            \}
        else \{
            sex <- 2
            sprows <- (spouselist[,2]==x & (spouselist[,4]!=spouselist[,3] |
                                            spouselist[,4] ==0))
            spouse <- spouselist[sprows, 1]
            \}
        \}
    # Marriages that cross levels are plotted at the higher level (lower
    #  on the paper).
    if (length(spouse)) \{
        keep <- level[spouse] <= lev
        spouse <- spouse[keep]
        sprows <- (which(sprows))[keep]
        \}
    nspouse <- length(spouse)  # Almost always 0, 1 or 2
\nwendcode{}\nwbegindocs{54}\nwdocspar
Create the set of 3 return structures, which will be matrices with
(1+nspouse) columns.
If there are children then other routines will widen the result.
\nwenddocs{}\nwbegincode{55}\moddef{alignped1}\plusendmoddef
    nid <- fam <- matrix(0, maxlev, nspouse+1)
    pos <- matrix(0.0, maxlev, nspouse +1)
    n[lev] <- nspouse +1       
    pos[lev,] <- 0:nspouse
    if (nspouse ==0) \{   
        # Easy case: the "tree rooted at x" is only x itself
        nid[lev,1] <- x
        return(list(nid=nid, pos=pos, fam=fam, n=n, spouselist=spouselist))
        \}
\nwendcode{}\nwbegindocs{56}\nwdocspar
Now we have a list of spouses that should be dealt with and 
the the correponding columns of the spouselist matrix.  
Create the two complimentary lists lspouse and rspouse to denote
those plotted on the left and on the right.  
For someone with lots of spouses we try to split them evenly.
If the number of spouses is odd, then men should have more on the
right than on the left, women more on the right.
Any hints in the spouselist matrix override.
We put the marriages whose direction is defined in the spouselist
closest to {\tt{}x}, then add undetermined ones to the left and
right in the order that they are found in the list.
The majority of marriages will be undetermined singletons, for which
nleft will be 1 for female (put my husband to the left) and 0 for male.
\nwenddocs{}\nwbegincode{57}\moddef{alignped1}\plusendmoddef
    lspouse <- spouse[spouselist[sprows,3] == 3-sex] # 1-2 or 2-1
    rspouse <- spouse[spouselist[sprows,3] == sex]   # 1-1 or 2-2
    if (any(spouselist[sprows,3] ==0)) \{
        #Not yet decided spouses
        indx <- which(spouselist[sprows,3] ==0)
        nleft <- floor((length(indx) + (sex==2))/2)
        if (nleft >0) \{
            lspouse <- c(spouse[indx[1:nleft]], lspouse)
            indx <- indx[-(1:nleft)]
          \}
        if (length(indx)) rspouse <- c(rspouse, spouse[indx])
      \}

    nid[lev,] <- c(lspouse, x, rspouse)
    nid[lev, 1:nspouse] <- nid[lev, 1:nspouse] + .5  #marriages    

    spouselist <- spouselist[-sprows,, drop=FALSE]
\nwendcode{}\nwbegindocs{58}\nwdocspar

The spouses are in the pedigree, now look below.
For each spouse get the list of children.
If there are any we call alignped2 to generate their tree and
then mark the connection to their parent.
If multiple marriages have children we need to join the
trees.
\nwenddocs{}\nwbegincode{59}\moddef{alignped1}\plusendmoddef
    nokids <- TRUE   #haven't found any kids yet
    spouse <- c(lspouse, rspouse)  #reorder
    for (i in 1:nspouse) \{
        ispouse <- spouse[i]
        children <- which((dad==x & mom==ispouse) | (dad==ispouse & mom==x))
        if (length(children) > 0) \{
            rval1 <- alignped2(children, dad, mom, level, horder, 
                              packed, spouselist)
            spouselist <- rval1$spouselist
            # set the parentage for any kids
            #  a nuisance: it's possible to have a child appear twice, when
            #  via inbreeding two children marry --- makes the "indx" line
            #  below more complicated
            temp <- floor(rval1$nid[lev+1,])  # cut off the .5's for matching
            indx <- (1:length(temp))[match(temp,children, nomatch=0) >0]
            rval1$fam[lev+1,indx] <- i   #set the kids parentage
            if (!packed) \{
                # line the kids up below the parents
                # The advantage at this point: we know that there is 
                #   nothing to the right that has to be cared for
                kidmean <- mean(rval1$pos[lev+1, indx])
                parmean <- mean(pos[lev, i + 0:1])
                if (kidmean > parmean) \{
                    # kids to the right of parents: move the parents
                    indx <- i:(nspouse+1)
                    pos[lev, indx] <- pos[lev, indx] + (kidmean - parmean)
                    \}
                else \{
                    # move the kids and their spouses and all below
                    shift <- parmean - kidmean
                    for (j in (lev+1):maxlev) \{
                        jn <- rval1$n[j]
                        if (jn>0) 
                            rval1$pos[j, 1:jn] <- rval1$pos[j, 1:jn] +shift
                        \}
                    \}
                \}
            if (nokids) \{
                rval <- rval1
                nokids <- FALSE
                \}
            else \{
                rval <- alignped3(rval, rval1, packed)
                \}
            \}
        \}
\nwendcode{}\nwbegindocs{60}\nwdocspar

To finish up we need to splice together the tree made up
from all the kids, which only has data from lev+1 down,
with the data here.  
There are 3 cases.  The first and easiest is when no
children were found.
The second, and most common, is when the tree below is
wider than the tree here, in which case we add the
data from this level onto theirs.
The third is when below is narrower, for instance an
only child.
\nwenddocs{}\nwbegincode{61}\moddef{alignped1}\plusendmoddef
    if (nokids) \{
        return(list(nid=nid, pos=pos, fam=fam, n=n, spouselist=spouselist))
        \}

    if (ncol(rval$nid) >= 1+nspouse) \{
        # The rval list has room for me!
        rval$n[lev] <- n[lev]
        indx <- 1:(nspouse+1)
        rval$nid[lev, indx] <- nid[lev,]
        rval$pos[lev, indx] <- pos[lev,]
        \}
    else \{
        #my structure has room for them
        indx <- 1:ncol(rval$nid)   
        rows <- (lev+1):maxlev
        n[rows] <- rval$n[rows]
        nid[rows,indx] <- rval$nid[rows,]
        pos[rows,indx] <- rval$pos[rows,]
        fam[rows,indx] <- rval$fam[rows,]
        rval <- list(nid=nid, pos=pos, fam=fam, n=n)
        \}
    rval$spouselist <- spouselist
    rval
    \}
\nwendcode{}\nwbegindocs{62}\nwdocspar

\subsection{alignped2}
This routine takes a collection of siblings, grows the tree for
each, and appends them side by side into a single tree.
The input arguments are the same as those to
{\tt{}alignped1} with the exception that {\tt{}x} will be a vector.
This routine does nothing to the spouselist matrix, but needs
to pass it down the tree and back since one of the routines
called by {\tt{}alignped2} might change the matrix.

The code below has one non-obvious special case.  Suppose
that two sibs marry.  
When the first sib is processed by {\tt{}alignped1} then both
partners (and any children) will be added to the rval
structure below.  
When the second sib is processed they
will come back as a 1 element tree (the marriage will no longer
be on the spouselist), which should \emph{not} be added
onto rval.  
The rule thus is to not add any 1 element tree whose
value (which must be x[i]) is already in the rval structure for this level.
(Where did Curtis O. \emph{find} these families?)

\nwenddocs{}\nwbegincode{63}\moddef{alignped2}\endmoddef
alignped2 <- function(x, dad, mom, level, horder, packed,
                      spouselist) \{
    x <- x[order(horder[x])]  # Use the hints to order the sibs
    rval <- alignped1(x[1],  dad, mom, level, horder, packed, 
                      spouselist)
    spouselist <- rval$spouselist

    if (length(x) >1) \{
        mylev <- level[x[1]]
        for (i in 2:length(x)) \{
            rval2 <-  alignped1(x[i], dad, mom, level,
                                horder, packed, spouselist)
            spouselist <- rval2$spouselist
            
            # Deal with the unusual special case:
            if ((rval2$n[mylev] > 1) || 
                          (is.na(match(x[i], floor(rval$nid[mylev,])))))
                rval <- alignped3(rval, rval2, packed)
            \}
        rval$spouselist <- spouselist
        \}
    rval
    \}
\nwendcode{}\nwbegindocs{64}\nwdocspar


\subsection{alignped3}
The third co-routine merges two pedigree trees which are side by
side into a single object.
The primary special case is when the rightmost person in the left
tree is the same as the leftmost person in the right tree; we 
needn't plot two copies of the same person side by side.
(When initializing the output structures don't worry about this - there
is no harm if they are a column bigger than finally needed.)
Beyond that the work is simple bookkeeping.

\nwenddocs{}\nwbegincode{65}\moddef{alignped3}\endmoddef
alignped3 <- function(x1, x2, packed, space=1) \{
    maxcol <- max(x1$n + x2$n)
    maxlev <- length(x1$n)
    n1 <- max(x1$n)   # These are always >1
    n  <- x1$n + x2$n

    nid <- matrix(0, maxlev, maxcol)
    nid[,1:n1] <- x1$nid
    
    pos <- matrix(0.0, maxlev, maxcol)
    pos[,1:n1] <- x1$pos

    fam <- matrix(0, maxlev, maxcol)
    fam[,1:n1] <- x1$fam
    fam2 <- x2$fam
    if (!packed) \{
        \LA{}align3-slide\RA{}
        \}
    \LA{}align3-merge\RA{}

    if (max(n) < maxcol) \{
        maxcol <- max(n)
        nid <- nid[,1:maxcol]
        pos <- pos[,1:maxcol]
        fam <- fam[,1:maxcol]
        \}

    list(n=n, nid=nid, pos=pos, fam=fam)
    \}
\nwendcode{}\nwbegindocs{66}\nwdocspar

For the unpacked case, which is the traditional way to draw a pedigree
when we can assume the paper is infinitely wide, all parents are centered
over their children.  
In this case we think if the two trees to be merged as solid blocks.
On input they both have a left margin of 0.
Compute how far over we have to slide the right tree.
\nwenddocs{}\nwbegincode{67}\moddef{align3-slide}\endmoddef
slide <- 0
for (i in 1:maxlev) \{
    n1 <- x1$n[i]
    n2 <- x2$n[i]
    if (n1 >0 & n2 >0) \{
        if (nid[i,n1] == x2$nid[i,1])
                temp <- pos[i, n1] - x2$pos[i,1]
        else    temp <- space + pos[i, n1] - x2$pos[i,1]
        if (temp > slide) slide <- temp
        \}
    \}
\nwendcode{}\nwbegindocs{68}\nwdocspar

Now merge the two trees. 
Start at the top level and work down.
\begin{enumerate}
  \item If n2=0, there is nothing to do
  \item Decide if there is a subject overlap, and if so 
    \begin{itemize}
      \item Set the proper parent id. 
        Only one of the two copies will be attached and the other
        will have fam=0, so max(fam, fam2) preserves the correct one.
      \item If not packed, set the position.  Choose the one connected
        to a parent, or midway for a double marriage.
    \end{itemize}
  \item If packed=TRUE determine the amount of slide for this row. It
    will be {\tt{}space} over from the last element in the left pedigree,
    less overlap.
  \item Move everything over
  \item Fix all the children of this level, right hand pedigree, to
    point to the correct parental position.
\end{enumerate}

\nwenddocs{}\nwbegincode{69}\moddef{align3-merge}\endmoddef
for (i in 1:maxlev) \{
    n1 <- x1$n[i]
    n2 <- x2$n[i]
    if (n2 >0) \{   # If anything needs to be done for this row...
        if (n1>0 && (nid[i,n1] == floor(x2$nid[i,1]))) \{
            #two subjects overlap
            overlap <- 1
            fam[i,n1] <- max(fam[i,n1], fam2[i,1])
            nid[i,n1] <- max(nid[i,n1], x2$nid[i,1]) #preserve a ".5"
            if (!packed) \{
                if(fam2[i,1]>0) 
                    if (fam[i,n1]>0) 
                        pos[i,n1] <- (x2$pos[i,1] + pos[i,n1] + slide)/2
                    else pos[i,n1] <- x2$pos[i,1]+ slide
                    \}
            n[i] <- n[i] -1
            \}
        else overlap <- 0
        
        if (packed) slide <- if (n1==0) 0 else pos[i,n1] + space - overlap

        zz <- seq(from=overlap+1, length=n2-overlap)
        nid[i, n1 + zz- overlap] <- x2$nid[i, zz]
        fam[i, n1 + zz -overlap] <- fam2[i,zz] 
        pos[i, n1 + zz -overlap] <- x2$pos[i,zz] + slide
        
        if (i<maxlev) \{
            # adjust the pointers of any children (look ahead)
            temp <- fam2[i+1,]
            fam2[i+1,] <- ifelse(temp==0, 0, temp + n1 -overlap)
            \}
        \}
    \}
\nwendcode{}\nwbegindocs{70}\nwdocspar

\section{alignped4}
The alignped4 routine is the final step of alignment.  It attempts to line
up children under parents and put spouses and siblings `close' to each other, 
to the extent possible within the constraints of page width.  This routine
used to be the most intricate and complex of the set, until I realized that
the task could be cast as constrained quadradic optimization.
The current code does necessary setup and then calls the {\tt{}quadprog}
function.  
At one point I investigated using one of the simpler least-squares routines
where $\beta$ is constrained to be non-negative. 
However a problem can only be translated into that form if the number
of constraints is less than the number of parameters, which is not
true in this problem.

There are two important parameters for the function.  One is the user specified
maximum width.  The smallest possible width is the maximum number of subjects
on a line, if the user's suggestion is too low it is increased to that 1+ that
amount (to give just a little wiggle room).
The other is a vector of 2 alignment parameters $a$ and $b$.
For each set of siblings ${x}$ with parents at $p_1$ and $p_2$ the
alignment penalty is
$$
   (1/k^a)\sum{i=1}{k} (x_i - (p_1 + p_2)^2
$$
where $k$ is the number of siblings in the set.
Using the fact that $\sum(x_i-c)^2 = \sum(x_i-\mu)^2 + k(c-\mu)^2$,
when $a=1$ then moving a sibship with $k$ sibs one unit to the left or
right of optimal will incur the same cost as moving one with only 1 or
two sibs out of place.  If $a=0$ then large sibships are harder to move
than small ones, with the default value $a=1.5$ they are slightly easier 
to move than small ones.  The rationale for the default is as long as the
parents are somewhere between the first and last siblings the result looks
fairly good, so we are more flexible with the spacing of a large family.
By tethering all the sibs to a single spot they tend are kept close to 
each other.
The alignment penalty for spouses is $b(x_1 - x_2)^2$, which tends to keep 
them together.  The size of $b$ controls the relative importance of sib-parent
and spouse-spouse closeness.

We start by adding in these penalties.  The total number of parameters
in the alignment problem (what we hand to quadprog) is the set 
of {\tt{}sum(n)} positions.  A work array myid keeps track of the parameter
number for each position so that it is easy to find.
There is one extra penalty added at the end.  Because the penalty amount
would be the same if all the final positions were shifted by a constant,
the penalty matrix will not be positive definite; solve.QP doesn't like
this.  We add a tiny amount of leftward pull to the widest line.
\nwenddocs{}\nwbegincode{71}\moddef{alignped4}\endmoddef
alignped4 <- function(rval, spouse, level, width, align) \{
    if (is.logical(align)) align <- c(1.5, 2)  #defaults
    maxlev <- nrow(rval$nid)
    width <- max(width, rval$n+.01)   # width must be > the longest row

    n <- sum(rval$n)  # total number of subjects
    myid <- matrix(0, maxlev, ncol(rval$nid))  #number the plotting points
    for (i in 1:maxlev) \{
        myid[i, rval$nid[i,]>0] <-  cumsum(c(0, rval$n))[i] + 1:rval$n[i]
        \}

    # There will be one penalty for each spouse and one for each child
    npenal <- sum(spouse[rval$nid>0]) + sum(rval$fam >0) 
    pmat <- matrix(0., nrow=npenal+1, ncol=n)

    indx <- 0
    # Penalties to keep spouses close
    for (lev in 1:maxlev) \{
        if (any(spouse[lev,])) \{
            who <- which(spouse[lev,])
            indx <- max(indx) + 1:length(who)
            pmat[cbind(indx, myid[lev,who])] <-  sqrt(align[2])
            pmat[cbind(indx, myid[lev,who+1])] <- -sqrt(align[2])
            \}
        \}

    # Penalties to keep kids close to parents
    for (lev in (1:maxlev)[-1])  \{ # no parents at the top level
        families <- unique(rval$fam[lev,])
        families <- families[families !=0]  #0 is the 'no parent' marker
        for (i in families) \{  #might be none
            who <- which(rval$fam[lev,] == i)
            k <- length(who)
            indx <- max(indx) +1:k   #one penalty per child
            penalty <- sqrt(k^(-align[1]))
            pmat[cbind(indx, myid[lev,who])] <- -penalty
            pmat[cbind(indx, myid[lev-1, rval$fam[lev,who]])] <- penalty/2
            pmat[cbind(indx, myid[lev-1, rval$fam[lev,who]+1])] <- penalty/2
            \}
        \}
    maxrow <- min(which(rval$n==max(rval$n)))
    pmat[nrow(pmat), myid[maxrow,1]] <- 1e-5
\nwendcode{}\nwbegindocs{72}\nwdocspar

Next come the constraints.  If there are $k$ subjects on a line there will
be $k+1$ constraints for that line.  The first point must be $\ge 0$, each
subesquent one must be at least 1 unit to the right, and the final point
must be $\le$ the max width.
\nwenddocs{}\nwbegincode{73}\moddef{alignped4}\plusendmoddef
    ncon <- n + maxlev    # number of constraints
    cmat <- matrix(0., nrow=ncon, ncol=n)
    coff <- 0  # cumulative constraint lines so var
    dvec <- rep(1., ncon)
    for (lev in 1:maxlev) \{
        nn <- rval$n[lev]
        if (nn>1) \{
            for (i in 1:(nn-1)) 
                cmat[coff +i, myid[lev,i + 0:1]] <- c(-1,1)
            \}

        cmat[coff+nn,   myid[lev,1]]  <- 1     #first element >=0
        dvec[coff+nn] <- 0
        cmat[coff+nn+1, myid[lev,nn]] <- -1    #last element <= width-1
        dvec[coff+nn+1] <- 1-width
        coff <- coff + nn+ 1
        \}

    if (exists('solve.QP')) \{
         pp <- t(pmat) %*% pmat + 1e-8 * diag(ncol(pmat))
         fit <- solve.QP(pp, rep(0., n), t(cmat), dvec)
         \}
    else stop("Need the quadprog package")

    newpos <- rval$pos
    #fit <- lsei(pmat, rep(0, nrow(pmat)), G=cmat, H=dvec)
    #newpos[myid>0] <- fit$X[myid]           
    newpos[myid>0] <- fit$solution[myid]
    newpos
    \}
\nwendcode{}\nwbegindocs{74}\nwdocspar
\section{Plots}
The plotting function for pedigrees has 5 tasks
\begin{enumerate}
  \item Gather information and check the data.  
    An important step is the call to align.pedigree.
  \item Set up the plot region and size the symbols.  
    The program wants to plot circles and squares,
    so needs to understand the geometry of the paper, pedigree size, and text 
    size to get the right shape and size symbols.
  \item Set up the plot and add the symbols for each subject
  \item Add connecting lines between spouses, and children with parents
  \item Create an invisible return value containing the locations.
\end{enumerate}
Another task, not yet completely understood, is how we might break a plot 
across multiple pages.

\nwenddocs{}\nwbegincode{75}\moddef{plot.pedigree}\endmoddef
plot.pedigree <- function(x, id = x$id, status = x$status, 
                          affected = x$affected, 
                          cex = 1, col = 1,
                          symbolsize = 1, branch = 0.6, 
                          packed = TRUE, align = c(1,2), width = 8, 
                          density=c(-1, 35,55,25), mar=c(4.1, 1, 4.1, 1),
                          angle=c(90,65,40,0), keep.par=FALSE,
                          subregion, ...)
\{
    Call <- match.call()
    n <- length(x$id)   
    \LA{}pedplot-data\RA{}
    \LA{}pedplot-sizing\RA{}
    \LA{}pedplot-symbols\RA{}
    \LA{}pedplot-lines\RA{}
    \LA{}pedplot-finish\RA{}
    \}
\nwendcode{}\nwbegindocs{76}\nwdocspar

\subsection{Setup}
The dull part is first: check all of the input data for
correctness.  
The {\tt{}sex} variable is taken from the pedigree so we need not check
that. 
The identifier for each subject is by default the {\tt{}id} variable from
the pedigree, but users often want to add some extra text.
The status variable can be used to put a line through the symbol
of those who are deceased, it is an optional part of the pedigree.
\nwenddocs{}\nwbegincode{77}\moddef{pedplot-data}\endmoddef
if(is.null(status))
  status <- rep(0, n)
else \{
    if(!all(status == 0 | status == 1))
      stop("Invalid status code")
    if(length(status) != n)
      stop("Wrong length for status")
\}
if(!missing(id)) \{
    if(length(id) != n)
      stop("Wrong length for id")
\}
\nwendcode{}\nwbegindocs{78}\nwdocspar
The ``affected status'' is a 0/1 matrix of any marker data that the
user might want to add.  It may be attached to the pedigree or added
here.  It can be a vector of length {\tt{}n} or a matrix with {\tt{}n} rows.
If it is not present, the default is to print open symbols without
shading or color, which corresponds to a code of 0, while a 1 means to
shade the symbol.  

If the argment is a matrix, then the shading and/or density value for
ith column is taken from the ith element of the angle/density arguments.

(Update by JPS 5/2011) Update to allow missing values (NA) in the ``affected''
indicators.  Missingness of affection status will have a ``?'' in 
the midpoint of the portion of the plot symbol rather than blank or shaded.
The ``?'' is in line with standards discussed in 
Bennet et a. J of Gent Couns., 2008.

For purposes within the plot method, NA values in ``affected'' are coded 
to -1.

\nwenddocs{}\nwbegincode{79}\moddef{pedplot-data}\plusendmoddef
if(is.null(affected))\{
  affected <- matrix(0,nrow=n)
\}
else \{
    if (is.matrix(affected))\{
        if (nrow(affected) != n) stop("Wrong number of rows in affected")
        if (is.logical(affected)) affected <- 1* affected
        if (ncol(affected) > length(angle) || ncol(affected) > length(density))
            stop("More columns in the affected matrix than angle/density values")
        \} 
    else \{
        if (length(affected) != n)
        stop("Wrong length for affected")

        if (is.logical(affected)) affected <- as.numeric(affected)
        if (is.factor(affected))  affected <- as.numeric(affected) -1
        \}
    if(max(affected, na.rm=TRUE) > min(affected, na.rm=TRUE)) \{
      affected <- matrix(affected - min(affected, na.rm=TRUE),nrow=n)
      affected[is.na(affected)] <- -1
    \} else \{
      affected <- matrix(affected,nrow=n)
    \}
    if (!all(affected==0 | affected==1 | affected == -1))
        stop("Invalid code for affected status")
\}

if (length(col) ==1) col <- rep(col, n)
else if (length(col) != n) stop("Col argument must be of length 1 or n")
\nwendcode{}\nwbegindocs{80}\nwdocspar

\subsection{Sizing}
Now we need to set the sizes. 
From align.pedigree we will get the maximum width and depth. 
There is one plotted row for each row of the returned matrices.
The number of columns of the matrices is the max width of the pedigree,
so there are unused positions in shorter rows, these can be identifed
by having an nid value of 0.
Horizontal locations for each point go from 0 to xmax, subjects are at
least 1 unit apart; a large number will be exactly one unit part.
These locations will be at the top center of each plotted symbol.
\nwenddocs{}\nwbegincode{81}\moddef{pedplot-sizing}\endmoddef
\LA{}pedplot-subregion\RA{}
plist <- align.pedigree(x, packed = packed, width = width, align = align)
if (!missing(subregion)) plist <- subregion2(plist, subset)
xrange <- range(plist$pos[plist$nid >0])
maxlev <- nrow(plist$pos)
\nwendcode{}\nwbegindocs{82}\nwdocspar

We would like to to make the boxes about 2.5 characters wide, which matches
most labels, but no more than 0.9 units wide or .5 units high.  
We also want to vertical room for the labels. Which should have at least
1/2 of stemp2 space above and stemp2 space below.  
The stemp3 variable is the height of labels: users may use multi-line ones.
Our constraints then are
\begin{itemize}
  \item (box height + label height)*maxlev $\le$ height: the boxes and labels have
    to fit vertically
  \item (box height) * (maxlev + (maxlev-1)/2) $\le$ height: at least 1/2 a box of
    space between each row of boxes
  \item (box width) $\le$ stemp1 in inches 
  \item (box width) $\le$ 0.8 unit in user coordinates, otherwise they appear 
    to touch
  \item User coordinates go from min(xrange)- 1/2 box width to 
    max(xrange) + 1/2 box width.
  \item the box is square (in inches)
\end{itemize}

The first 3 of these are easy.  The fourth comes into play only for very packed
pedigrees. Assume that the box were the maximum size of .8 units, i.e., minimal
spacing between them. Then xmin -.45 to xmax + .45 covers the plot region,
the scaling between user coordinates and inches is (.8 + xmax-xmin) user =
figure region inches, and the box is .8*(figure width)/(.8 + xmax-xmin).
The transformation from user units to inches horizontally depends on the box
size, since I need to allow for 1/2 a box on the left and right.  
Vertically the range from 1 to nrow spans the tops of the symbols, which 
will be the figure region height less (the height of the
text for the last row + 1 box); remember that the coordinates point to the
top center of the box.
We want row 1 to plot at the top, which is done by appropriate setting
of the usr parameter.
\nwenddocs{}\nwbegincode{83}\moddef{pedplot-sizing}\plusendmoddef
frame()
oldpar <- par(mar=mar, xpd=TRUE)
psize <- par('pin')  # plot region in inches
stemp1 <- strwidth("ABC", units='inches', cex=cex)* 2.5/3
stemp2 <- strheight('1g', units='inches', cex=cex)
stemp3 <- max(strheight(id, units='inches', cex=cex))

ht1 <- psize[2]/maxlev - (stemp3 + 1.5*stemp2)
if (ht1 <=0) stop("Labels leave no room for the graph, reduce cex")
ht2 <- psize[2]/(maxlev + (maxlev-1)/2)
wd2 <- .8*psize[1]/(.8 + diff(xrange))

boxsize <- symbolsize* min(ht1, ht2, stemp1, wd2) # box size in inches
hscale <- (psize[1]- boxsize)/diff(xrange)  #horizontal scale from user-> inch
vscale <- (psize[2]-(stemp3 + stemp2/2 + boxsize))/ max(1, maxlev-1)
boxw  <- boxsize/hscale  # box width in user units
boxh  <- boxsize/vscale   # box height in user units
labh  <- stemp2/vscale   # height of a text string
legh  <- min(1/4, boxh  *1.5)  # how tall are the 'legs' up from a child
par(usr=c(xrange[1]- boxw/2, xrange[2]+ boxw/2, 
          maxlev+ boxh+ stemp3 + stemp2/2 , 1))
\nwendcode{}\nwbegindocs{84}\nwdocspar

\subsection{Drawing the tree}
Now we draw and label the boxes.  Definition of the drawbox function is
deferred until later.
\nwenddocs{}\nwbegincode{85}\moddef{pedplot-symbols}\endmoddef
\LA{}pedplot-drawbox\RA{}

sex <- as.numeric(x$sex)
for (i in 1:maxlev) \{
    for (j in 1:plist$n[i]) \{
        k <- plist$nid[i,j]
        drawbox(plist$pos[i,j], i, sex[k], affected[k,],
                status[k], col[k], polylist, density, angle,
                boxw, boxh)
        text(plist$pos[i,j], i + boxh + labh*.7, id[k], cex=cex, adj=c(.5,1))
        \}
\}
\nwendcode{}\nwbegindocs{86}\nwdocspar

Now draw in the connections one by one. First those between spouses.
\nwenddocs{}\nwbegincode{87}\moddef{pedplot-lines}\endmoddef
maxcol <- ncol(plist$nid)  #all have the same size
for(i in 1:maxlev) \{
    tempy <- i + boxh/2
    if(any(plist$spouse[i,  ]>0)) \{
        temp <- (1:maxcol)[plist$spouse[i,  ]>0]
        segments(plist$pos[i, temp] + boxw/2, rep(tempy, length(temp)), 
             plist$pos[i, temp + 1] - boxw/2, rep(tempy, length(temp)))

        temp <- (1:maxcol)[plist$spouse[i,  ] ==2]
        if (length(temp)) \{ #double line for double marriage
            tempy <- tempy + boxh/10
            segments(plist$pos[i, temp] + boxw/2, rep(tempy, length(temp)), 
               plist$pos[i, temp + 1] - boxw/2, rep(tempy, length(temp)))
            \}
    \}
\}
\nwendcode{}\nwbegindocs{88}\nwdocspar
Now connect the children to the parents.  First there are lines up from each
child, which would be trivial except for twins, triplets, etc.  Then we 
draw the horizontal bar across siblings and finally the connector from
the parent.  For twins, the ``vertical'' lines are angled towards a 
common point, the variable is called {\tt{}target} below.
The horizontal part is easier if we do things family by
family.  The {\tt{}plist{\char36}twins} variable is 1/2/3 for a twin on my right,
0 otherwise.

\nwenddocs{}\nwbegincode{89}\moddef{pedplot-lines}\plusendmoddef
for(i in 2:maxlev) \{
    zed <- unique(plist$fam[i,  ])
    zed <- zed[zed > 0]  #list of family ids
    
    for(fam in zed) \{
        xx <- plist$pos[i - 1, fam + 0:1]
        parentx <- mean(xx)   #midpoint of parents


        # Draw the uplines
        who <- (plist$fam[i,] == fam) #The kids of interest
        if (is.null(plist$twins)) target <- plist$pos[i,who]
        else \{
            twin.to.left <-(c(0, plist$twins[i,who])[1:sum(who)])
            temp <- cumsum(twin.to.left ==0) #increment if no twin to the left
            # 5 sibs, middle 3 are triplets gives 1,2,2,2,3
            # twin, twin, singleton gives 1,1,2,2,3
            tcount <- table(temp)
            target <- rep(tapply(plist$pos[i,who], temp, mean), tcount)
            \}
        yy <- rep(i, sum(who))
        segments(plist$pos[i,who], yy, target, yy-legh)
                  
        ## draw midpoint MZ twin line
        if (any(plist$twins[i,who] ==1)) \{
          who2 <- which(plist$twins[i,who] ==1)
          temp1 <- (plist$pos[i, who][who2] + target[who2])/2
          temp2 <- (plist$pos[i, who][who2+1] + target[who2])/2
            yy <- rep(i, length(who2)) - legh/2
            segments(temp1, yy, temp2, yy)
            \}

        # Add a question mark for those of unknown zygosity
        if (any(plist$twins[i,who] ==3)) \{
          who2 <- which(plist$twins[i,who] ==3)
          temp1 <- (plist$pos[i, who][who2] + target[who2])/2
          temp2 <- (plist$pos[i, who][who2+1] + target[who2])/2
            yy <- rep(i, length(who2)) - legh/2
            text((temp1+temp2)/2, yy, '?')
            \}
        
        # Add the horizontal line 
        segments(min(target), i-legh, max(target), i-legh)

        # Draw line to parents
        x1 <- mean(range(target))
        y1 <- i-legh
        if(branch == 0)
            segments(x1, y1, parentx, (i-1) + boxh/2)
        else \{
            y2 <- (i-1) + boxh/2
            x2 <- parentx
            ydelta <- ((y2 - y1) * branch)/2
            segments(c(x1, x1, x2), c(y1, y1 + ydelta, y2 - ydelta), 
                     c(x1, x2, x2), c(y1 + ydelta, y2 - ydelta, y2))
            \}
        \}
    \}
\nwendcode{}\nwbegindocs{90}\nwdocspar

The last set of lines are dotted arcs that connect mulitiple instances of
a subject on the same line.  These instances may or may not be on the
same line.
The arrcconect function draws a quadratic arc between locations $(x_1, y_1)$
and $(x_2, y_2$) whose height is 1/2 unit above a straight line connection.
\nwenddocs{}\nwbegincode{91}\moddef{pedplot-lines}\plusendmoddef
arcconnect <- function(x, y) \{
    xx <- seq(x[1], x[2], length = 15)
    yy <- seq(y[1], y[2], length = 15) + (seq(-7, 7))^2/98 - .5
    lines(xx, yy, lty = 2)
    \}

uid <- unique(plist$nid)
for (id in uid[uid>0]) \{
    indx <- which(plist$nid == id)
    if (length(indx) >1) \{   #subject is a multiple
        tx <- plist$pos[indx]
        ty <- ((row(plist$pos))[indx])[order(tx)]
        tx <- sort(tx)
        for (j in 1:(length(indx) -1))
            arcconnect(tx[j + 0:1], ty[j+  0:1])
        \}
    \}
\nwendcode{}\nwbegindocs{92}\nwdocspar

\subsection{Final output}
Remind the user of subjects who did not get
plotted; these are ususally subjects who are married in but without
children.  Unless the pedigree contains spousal information the
routine does not know who is the spouse.
Then restore the plot parameters.  This would only not be done if someone
wants to further annotate the plot.
Last, we give a list of the plot positions for each subject.  Someone
who is plotted twice will have their first position listed.
\nwenddocs{}\nwbegincode{93}\moddef{pedplot-finish}\endmoddef
ckall <- x$id[is.na(match(x$id,x$id[plist$nid]))]
if(length(ckall>0)) cat('Did not plot the following people:',ckall,'\\n')
    
if(!keep.par) par(oldpar)

tmp <- match(unique(plist$nid[plist$nid!=0]), plist$nid)
invisible(list(plist=plist, x=plist$pos[tmp], y= -row(plist$pos)[tmp],
               boxw=boxw, boxh=boxh, call=Call))        
\nwendcode{}\nwbegindocs{94}\nwdocspar
\subsection{Symbols}
There are four sumbols corresponding to the four sex codes: square = male,
circle = female, diamond= unknown, and triangle = terminated.  
They are shaded according to the value(s) of affected status for each
subject, where 0=unfilled and 1=filled, and filling uses the standard
arguments of the {\tt{}polygon} function.
The nuisance is when the affected status is a matrix, in which case the
symbol will be divided up into sections, clockwise starting at the 
lower left. 
I asked Beth about this (original author) and there was no particular
reason to start at 6 o'clock, but it's now established as history.

The first part of the code is to create the collection of polygons that
will make up the symbol.  These are then used again and again.
The collection is kept as a list with the four elements square, circle,
diamond and triangle.  
Each of these is in turn a list with ncol(affected) element, and each
of those in turn a list of x and y coordinates.
There are 3 cases: the affected matrix has
only one column, partitioning a circle for multiple columns, and 
partitioning the other cases for multiple columns.

\nwenddocs{}\nwbegincode{95}\moddef{pedplot-drawbox}\endmoddef
\LA{}pedplot-circfun\RA{}
\LA{}pedplot-polyfun\RA{}
if (ncol(affected)==1) \{
    polylist <- list(
        square = list(list(x=c(-1, -1, 1,1)/2,  y=c(0, 1, 1, 0))),
        circle = list(list(x=.5* cos(seq(0, 2*pi, length=50)),
                           y=.5* sin(seq(0, 2*pi, length=50)) + .5)),
        diamond = list(list(x=c(0, -.5, 1, .5), y=c(0, .5, 1, .5))),
        triangle= list(list(x=c(0, -.56, .56),  y=c(0, 1, 1))))
    \}
else \{
    nc <- ncol(affected)
    square <- polyfun(nc, list(x=c(-.5, -.5, .5, .5), y=c(-.5, .5, .5, -.5),
                                theta= -c(3,5,7,9)* pi/4))
    circle <- circfun(nc)
    diamond <-  polyfun(nc, list(x=c(0, -.5, 0, .5), y=c(-.5, 0, .5,0),
                                theta= -(1:4) *pi/2))
    triangle <- polyfun(nc, list(x=c(-.56, .0, .56), y=c(-.5, .5, -.5),
                                 theta=c(-2, -4, -6) *pi/3))
    polylist <- list(square=square, circle=circle, diamond=diamond, 
                     triangle=triangle)
    \}
\nwendcode{}\nwbegindocs{96}\nwdocspar

The circle function is quite simple.  The number of segments is arbitrary,
50 seems to be enough to make the eye happy.  We draw the ray from 0 to
the edge, then a portion of the arc.  The polygon function will connect
back to the center.
\nwenddocs{}\nwbegincode{97}\moddef{pedplot-circfun}\endmoddef
circfun <- function(nslice, n=50) \{
    nseg <- ceiling(n/nslice)  #segments of arc per slice
    
    theta <- -pi/2 - seq(0, 2*pi, length=nslice +1)
    out <- vector('list', nslice)
    for (i in 1:nslice) \{
        theta2 <- seq(theta[i], theta[i+1], length=nseg)
        out[[i]]<- list(x=c(0, cos(theta2)/2),
                        y=c(0, sin(theta2)/2) + .5)
        \}
    out
    \}
\nwendcode{}\nwbegindocs{98}\nwdocspar

Now for the interesting one --- dividing a polygon into ``pie slices''.
In computing this we can't use the usual $y= a + bx$ formula for a line,
because it doesn't work for vertical ones (like the sides of the square).
Instead we use the alternate formulation in terms of a dummy variable 
$z$.
\begin{eqnarray*}
  x &=& a + bz \\
  y &=& c + dz \\
\end{eqnarray*}
Furthermore, we choose the constants $a$, $b$, $c$, and $d$ so that 
the side of our polygon correspond to $0 \le z \le 1$.
The intersection of a particular ray at angle theta with a 
particular side will satisfy
\begin{eqnarray}
  theta &=& y/x = \frac{a + bz}{c+dz} \nonumber \\
  z &=& \frac{a\theta -c}{b - d\theta} \label{eq:z} \\
\end{eqnarray}

Equation \ref{eq:z} will lead to a division by zero if the ray from the
origin does not intersect a side, e.g., a vertical divider will be parallel
to the sides of a square symbol.  The only solutions we want have
$0 \le z \le 1$ and are in the `forward' part of the ray.  This latter  %'`
is true if the inner product $x \cos(\theta) + y \sin(\theta) >0$.
Exactly one of the polygon sides will satisfy both conditions.

\nwenddocs{}\nwbegincode{99}\moddef{pedplot-polyfun}\endmoddef
polyfun <- function(nslice, object) \{
    # make the indirect segments view
    zmat <- matrix(0,ncol=4, nrow=length(object$x))
    zmat[,1] <- object$x
    zmat[,2] <- c(object$x[-1], object$x[1]) - object$x
    zmat[,3] <- object$y
    zmat[,4] <- c(object$y[-1], object$y[1]) - object$y

    # Find the cutpoint for each angle
    #   Yes we could vectorize the loop, but nslice is never bigger than
    # about 10 (and usually <5), so why be obscure?
    ns1 <- nslice+1
    theta <- -pi/2 - seq(0, 2*pi, length=ns1)
    x <- y <- double(ns1)
    for (i in 1:ns1) \{
        z <- (tan(theta[i])*zmat[,1] - zmat[,3])/
            (zmat[,4] - tan(theta[i])*zmat[,2])
        tx <- zmat[,1] + z*zmat[,2]
        ty <- zmat[,3] + z*zmat[,4]
        inner <- tx*cos(theta[i]) + ty*sin(theta[i])
        indx <- which(is.finite(z) & z>=0 &  z<=1 & inner >0)
        x[i] <- tx[indx]
        y[i] <- ty[indx]
        \}
\nwendcode{}\nwbegindocs{100}\nwdocspar

Now I have the $x,y$ coordinates where each radial slice (the cuts you
would make when slicing a pie) intersects the polygon.  
Add the original vertices of the polygon to the list, sort by angle, and
create the output.  The radial lines are labeled 1,2, \ldots, nslice +1
(the original cut from the center to 6 o'clock is repeated at the end),   %'
and the inserted vertices with a zero.
\nwenddocs{}\nwbegincode{101}\moddef{pedplot-polyfun}\plusendmoddef
    nvertex <- length(object$x)
    temp <- data.frame(indx = c(1:ns1, rep(0, nvertex)),
                       theta= c(theta, object$theta),
                       x= c(x, object$x),
                       y= c(y, object$y))
    temp <- temp[order(-temp$theta),]
    out <- vector('list', nslice)
    for (i in 1:nslice) \{
        rows <- which(temp$indx==i):which(temp$indx==(i+1))
        out[[i]] <- list(x=c(0, temp$x[rows]), y= c(0, temp$y[rows]) +.5)
        \}
    out
    \}   
\nwendcode{}\nwbegindocs{102}\nwdocspar

Finally we get to the drawbox function itself, which is fairly simple.
Updates by JPS in 5/2011 to allow missing, and to fix up shadings and borders.
For affected=0, don't fill.
For affected=1, fill with density-lines and angles.
For affected=-1 (missing), fill with ``?'' in the midpoint of the polygon,
with a size adjusted by how many columns in affected.
For all shapes drawn, make the border the color for the person.

\nwenddocs{}\nwbegincode{103}\moddef{pedplot-drawbox}\plusendmoddef

  drawbox<- function(x, y, sex, affected, status, col, polylist,
            density, angle, boxw, boxh) \{
        for (i in 1:length(affected)) \{
            if (affected[i]==0) \{
                polygon(x + (polylist[[sex]])[[i]]$x *boxw,
                        y + (polylist[[sex]])[[i]]$y *boxh,
                        col=NA, border=col)
                \}
            
            if(affected[i]==1) \{
              ## else \{
              polygon(x + (polylist[[sex]])[[i]]$x * boxw,
                      y + (polylist[[sex]])[[i]]$y * boxh,
                      col=col, border=col, density=density[i], angle=angle[i])            
            \}
            if(affected[i] == -1) \{
              polygon(x + (polylist[[sex]])[[i]]$x * boxw,
                      y + (polylist[[sex]])[[i]]$y * boxh,
                      col=NA, border=col)
              
              midx <- x + mean(range(polylist[[sex]][[i]]$x*boxw))
              midy <- y + mean(range(polylist[[sex]][[i]]$y*boxh))
             
              points(midx, midy, pch="?", cex=min(1, cex*2/length(affected)))
            \}
            
          \}
        if (status==1) segments(x- .6*boxw, y+1.1*boxh, 
                                x+ .6*boxw, y- .1*boxh,)
        ## Do a black slash per Beth, old line was
        ##        x+ .6*boxw, y- .1*boxh, col=col)
      \}

\nwendcode{}\nwbegindocs{104}\nwdocspar

\subsection{Subsetting}
This section is still experimental and might change.  

Sometimes a pedigree is too large to fit comfortably on one page.
The {\tt{}subset} argument allows one to plot only a portion of the
pedigree based on the plot region.  Along with other tools to
select portions of the pedigree based on relatedness, such as all
the descendents of a particular marriage, it gives a tool for
addressing this.  This breaks our original goal of completely
automatic plots, but users keep asking for more.

The argument is {\tt{}subregion=c(min\ x,\ max\ x,\ min\ depth,\ max\ depth)},
and works by editing away portions of the {\tt{}plist} object
returned by align.pedigree. 
First decide what lines to keep. 
Then take subjects away from each line, 
update spouses and twins,
and fix up parentage for the line below.
\nwenddocs{}\nwbegincode{105}\moddef{pedplot-subregion}\endmoddef
subregion2 <- function(plist, subset) \{
    if (subset[3] <1 || subset[4] > length(plist$n)) 
        stop("Invalid depth indices in subset")
    lkeep <- subset[3]:subset[4]
    for (i in lkeep) \{
        if (!any(plist$pos[i,]>=subset[1] & plist$pos[i,] <= subset[2]))
            stop(paste("No subjects retained on level", i))
        \}
    
    nid2 <- plist$nid[lkeep,]
    n2   <- plist$n[lkeep]
    pos2 <- plist$pos[lkeep,]
    spouse2 <- plist$spouse[lkeep,]
    fam2 <- plist$fam[lkeep,]
    if (!is.null(plist$twins)) twin2 <- plist$twins[lkeep,]
    
    for (i in 1:nrow(nid2)) \{
        keep <- which(pos2[i,] >=subset[1] & pos2[i,] <= subset[2])
        nkeep <- length(keep)
        n2[i] <- nkeep
        nid2[i, 1:nkeep] <- nid2[i, keep]
        pos2[i, 1:nkeep] <- pos2[i, keep]
        spouse2[i,1:nkeep] <- spouse2[i,keep]
        fam2[i, 1:nkeep] <- fam2[i, keep]
        if (!is.null(plist$twins)) twin2[i, 1:nkeep] <- twin2[i, keep]

        if (i < nrow(nid2)) \{  #look ahead
            tfam <- match(fam2[i+1,], keep, nomatch=0)
            fam2[i+1,] <- tfam
            if (any(spouse2[i,tfam] ==0)) 
                stop("A subset cannot separate parents")
            \}
        \}
    
    n <- max(n2)
    out <- list(n= n2[1:n], nid=nid2[,1:n, drop=F], pos=pos2[,1:n, drop=F],
                spouse= spouse2[,1:n, drop=F], fam=fam2[,1:n, drop=F])
    if (!is.null(plist$twins)) out$twins <- twin2[, 1:n, drop=F]
    out
    \}
\nwendcode{}\nwbegindocs{106}\nwdocspar
\documentclass{article}
\usepackage{noweb}
\usepackage[pdftex]{graphicx}
%\usepackage{times}
\addtolength{\textwidth}{1in}
\addtolength{\oddsidemargin}{-.5in}
\setlength{\evensidemargin}{\oddsidemargin}

\newcommand{\myfig}[1]{\resizebox{\textwidth}{!}
                        {\includegraphics{figure/#1.pdf}}}

\noweboptions{breakcode}
\title{The \emph{pedigree.shrink} functions in R}
\author{Daniel Schaid, Shannon McDonnell, Steven Iturria, Erin Carlson, Jason Sinnwell}

\begin{document}
\maketitle
\section{Intro to Pedigree Shrink}

[Written by Jason Sinnwell without double-checking many of the facts]

The pedigree.shrink functions were initially written to deal with a pedigree
represented as a data.frame with pedTrim, written by Steve Iturria, to trim 
the subjects from a pedigree who were less useful for linkage and family 
association studies.  It was later turned into a package called pedShrink 
by Daniel Schaid's group, still working on a pedigree, but assuming it was 
just a data.frame.  Later, the functions were managed by Jason Sinnwell who 
worked with the 2010 version of the pedigree object by Terry Therneau in 
planning to group many of the pedigree functions together into an enhanced 
kinship package.

This file also contains the pedigree.unrelated function, developed by Dan 
Schaid and Shannon McDonnell, which uses the kinship matrix to 
determine relatedness of subjects in a pedigree, and returns the person id
of the individuals that are less related. Details described below.


\section{Pedigree Shrink}
The pedigree.shrink function trims an object of class pedigree, and returns a list with information about how the pedigree was shrunk, and the final shrunken pedigree object.
\emph{pedigree.shrink}.  
Accepts the following input
\begin{description}
  \item[ped] a pedigree object
  \item[avail] indicator vector of availability of each person in the pedigree
  \item[seed] seed to control randomness
  \item[maxBits] bit size to shrink the pedigree size under
\end{description}


\nwenddocs{}\nwbegincode{107}\moddef{pedigree.shrink}\endmoddef
#$Log: pedigree.shrink.q,v $
#Revision 1.5  2010/09/03 21:11:16  sinnwell
#add shrunk "avail" vector to result, keep status and affected in pedObj
#
#Revision 1.4  2010/09/03 19:15:03  sinnwell
#add avail arg which is not part of ped object.  Re-make ped object at the end with status and affected, if given
#
#Revision 1.2  2009/11/17 23:08:18  sinnwell
#*** empty log message ***
#
#Revision 1.1  2008/07/16 20:23:07  sinnwell
#Initial revision
#

pedigree.shrink <- function(ped, avail, affected=NULL, seed=NULL, maxBits = 16)\{
  
  ## set the seed for random selections
  if(is.null(seed))
    \{
      seed <- sample(2^20, size=1)
    \}
  set.seed(seed)

  if(any(is.na(avail)))
    stop("NA values not allowed in avail vector.")
  
  if(is.null(affected))
    affected = if(is.matrix(ped$affected)) ped$affected[,1] else ped$affected
  
  
  ## create dataframe from pedigree object, with avail added on
  ## then re-make pedigree object at the end after shrunk
  pedData <- data.frame(id=ped$id, father=ped$findex, mother=ped$mindex, 
                 sex=ped$sex, affected=affected, 
                 status=if(length(ped$status)) ped$status else rep(0, length(ped$id)),
                 avail=as.logical(avail))
  
  idTrimmed <- numeric()
  nOriginal <- nrow(pedData)
  bitSizeOriginal <- pedBits(pedData$id, pedData$father, pedData$mother)$bitSize
  
  ## first run pedTrim to remove anyone who is not available and
  ## does not have an available descendant
  datTrim <- pedTrim(pedData)
  if(nrow(datTrim) < nrow(pedData)) 
    idTrimmed <- c(idTrimmed, pedData$id[is.na(match(pedData$id, datTrim$id))])

  
  ## remove from pre-trimmed data info that can change by trimming,
  ## to be replaced by the post-trim data 
  pedData <- pedData[,c("id", "sex", "affected", "status")]
  pedData <- merge(datTrim, pedData)

 
  ## now trim any available terminal subjects with unknown phenotype
  ## but only if both parents are available
  nChange <- 1


 while(nChange > 0)\{
    nOld <- nrow(pedData)
    datTrim <- trimAvailNonInform(pedData)
    
    pedData <- pedData[,c("id", "sex", "affected", "status")]
    
    if(nrow(datTrim) < nrow(pedData)) 
      idTrimmed <- c(idTrimmed, pedData$id[is.na(match(pedData$id, datTrim$id))])
    
    pedData <- merge(datTrim, pedData)
    nNew <- nrow(pedData)
    nChange <-  nOld - nNew
  \}

  
  ##  Determine number of subjects & bitSize after initial trimming
  nIntermed <- nrow(pedData)
  bitSize <- pedBits(pedData$id, pedData$father, pedData$mother)$bitSize
  
  ## now begin to shrink to fit bitSize <= maxBits

  bitVec <- c(bitSizeOriginal,bitSize)
  
  isTrimmed <- TRUE
  
  while(isTrimmed & (bitSize > maxBits))
    \{  
      
      ## First, try trimming by unknown status
      save <- pedTrimOneSubj(pedData, affstatus = NA)    
      isTrimmed <- save$isTrimmed
      
      ## Second, try trimming by unaffected status if no unknowns to trim
      if(!isTrimmed)
        \{
          save <- pedTrimOneSubj(pedData, affstatus = 0)
          isTrimmed <- save$isTrimmed
        \}
      
      
      ## Third, try trimming by affected status if no unknowns & no unaffecteds
      ## to trim
      if(!isTrimmed) \{
          save <- pedTrimOneSubj(pedData, affstatus = 1)
          isTrimmed <- save$isTrimmed
      \}
      
      if(isTrimmed)  \{
          pedData <- save$pedData
          bitSize <- save$bitSize
          bitVec <- c(bitVec, bitSize)          
          idTrimmed <- c(idTrimmed, save$idTrimmed)
        \}   
  
    \}
  
  nFinal <- nrow(pedData)
  
  ## make shrunk pedigree object
  pedObj <- pedigree(pedData$id, pedData$father, pedData$mother,
                     pedData$sex, affected=pedData$affected, 
                     status=pedData$status)
  
  obj <- list(pedObj = pedObj,
              idTrimmed = idTrimmed,
              bitSize = bitVec,
              avail=avail[match(pedObj$id, ped$id)],
              pedSizeOriginal = nOriginal,
              pedSizeIntermed = nIntermed,
              pedSizeFinal  = nFinal,
              seed = seed)
  
  oldClass(obj) <- "pedigree.shrink"
  
  return(obj)
\}


\nwendcode{}\nwbegindocs{108}\nwdocspar



\subsection{Sub-Functions}


These are functions used within pedigree.shrink

pedTrim: remove from ped file anyone who is not available and does not have 
an available descendant.  Do this iteratively by successively removing unavailable 
terminal nodes.  Written by  Steve Iturria, PhD, modified by Dan Schaid.
\nwenddocs{}\nwbegincode{109}\moddef{pedTrim}\endmoddef

#$Log: pedTrim.q,v $
#Revision 1.4  2009/11/19 15:00:31  sinnwell
#*** empty log message ***
#
#Revision 1.3  2009/11/19 14:57:05  sinnwell
#*** empty log message ***
#
#Revision 1.2  2009/11/17 23:11:09  sinnwell
#change for ped object
#
#Revision 1.1  2008/07/16 20:23:29  sinnwell
#Initial revision
#

pedTrim <-function(pedData) \{

  ## remove from ped file anyone who is not available and
  ## does not have an available descendant
  
  ## avail = TRUE if available, FALSE if not
  
  ## will do this iteratively by successively removing unavailable
  ## terminal nodes
  ## Steve Iturria, PhD, modified by Dan Schaid
  
  cont <- TRUE                  # flag for whether to keep iterating
  
  is.terminal <- (is.parent(pedData$id, pedData$father, pedData$mother) == FALSE
)
  pedData <- cbind.data.frame(pedData, is.terminal)
  
 
  iter <- 1

  while(cont)  \{
    ##print(paste("Working on iter", iter))
    
    num.found <- 0
    idx.to.remove <- NULL
    
    for(i in 1:nrow(pedData))
      \{
        
        if( pedData$is.terminal[i] )
          \{
            if( pedData$avail[i] == FALSE )   # if not genotyped         
              \{
                idx.to.remove <- c(idx.to.remove, i)
                num.found <- num.found + 1
                
                ## print(paste("  removing", num.found, "of", nrow(pedData)))
              \}
          \}
      \}
    
    if(num.found > 0) \{
      pedData <- pedData[-idx.to.remove, ]
      pedData$is.terminal <-  
        (is.parent(pedData$id, pedData$father, pedData$mother) == FALSE)
    \}
    else \{
      cont <- FALSE
    \}
    iter <- iter + 1
  \}
  
  
  ## a few more clean up steps
  tmpPed <- excludeUnavailFounders(pedData$id, pedData$father, pedData$mother, pedData$avail)
  tmpPed <- excludeStrayMarryin(tmpPed$id, tmpPed$father, tmpPed$mother)
  
  ## get rid of parents, to replace them via merge with new parental codes that  might
  ## be zeroed out due to removal of unavailable founders
  
  pedData$father <- NULL  
  pedData$mother <- NULL
  pedData <- merge(tmpPed, pedData)
  pedData$is.terminal <- NULL
  
  return(pedData)
  
\}



\nwendcode{}\nwbegindocs{110}\nwdocspar


Group other functions used in the above main functions
together as pedigree.shrink.minor.R

\nwenddocs{}\nwbegincode{111}\moddef{pedigree.shrink.minor}\endmoddef

#$Log: pedigree.shrink.minor.q,v $
#Revision 1.5  2009/11/19 18:10:26  sinnwell
#F to FALSE
#
#Revision 1.4  2009/11/19 14:57:13  sinnwell
#*** empty log message ***
#
#Revision 1.3  2009/11/17 23:11:41  sinnwell
#*** empty log message ***
#
#Revision 1.1  2008/07/16 20:22:55  sinnwell
#Initial revision
#

pedBits <- function(id, father, mother)\{

  founder <- father==0 & mother==0
  pedSize <- length(father)
  nFounder <- sum(founder)
  nNonFounder <- pedSize - nFounder
  bitSize <- 2*nNonFounder - nFounder
  return(list(bitSize=bitSize,
              nFounder = nFounder,
              nNonFounder = nNonFounder))

\}

\nwendcode{}\nwbegindocs{112}\nwdocspar


Remove a subject from a pedigree-like data.frame

\nwenddocs{}\nwbegincode{113}\moddef{pedigree.shrink.minor}\plusendmoddef
 
removeSubj <- function(removeId, pedData)\{
  tmp.avail  <- pedData$avail
  tmp.avail[pedData$id == removeId] <- FALSE
  pedData$avail <- tmp.avail
  newPed <- pedTrim(pedData)
  pedData <- pedData[,c('id','sex','affected')]
  newPed <- merge(newPed, pedData)

  return(newPed)
\}

\nwendcode{}\nwbegindocs{114}\nwdocspar

Get indicator vector of who is a parent, founder, or disconnected

\nwenddocs{}\nwbegincode{115}\moddef{pedigree.shrink.minor}\plusendmoddef
is.parent <- function(id, father, mother)\{
  # determine subjects who are parents

  isFather <- !is.na(match(id, unique(father[father!=0])))
  isMother <- !is.na(match(id, unique(mother[mother!=0])))
  isParent <- isFather |isMother
  return(isParent)
\}

is.founder <- function(mother, father)\{
  check <- (father==0) & (mother==0)
  return(check)
\}


is.disconnected <- function(id, father, mother)
\{
  # check to see if any subjects are disconnected in pedigree by checking for
  # kinship = 0 for all subjects excluding self

  kinMat <- kinship(save$pedData$id, save$pedData$father, save$pedData$mother)
  diag(kinMat) <- 0
  disconnected <- apply(kinMat==0.0, 1, all)

  return(disconnected)
\}


\nwendcode{}\nwbegindocs{116}\nwdocspar

Try trimming one subject by with affected matching affstatus.  If there are
ties of multiple subjects that reduce bit size equally, randomly choose one of them.

\nwenddocs{}\nwbegincode{117}\moddef{pedigree.shrink.minor}\plusendmoddef

pedTrimOneSubj <- function(pedData, affstatus)
  ## Try trimming one subject by affection status level. 
  ## If ties for bits removed, randomly select one of the ids
  \{
 
    notParent <- !is.parent(pedData$id, pedData$father, pedData$mother)
    if(is.na(affstatus)) \{
      possiblyTrim <- pedData$id[notParent & pedData$avail & is.na(pedData$affected)]
    \} else \{
      possiblyTrim <- pedData$id[notParent & pedData$avail & pedData$affected==affstatus]
    \}
    nTrim <- length(possiblyTrim)
    
    if(nTrim == 0)
      \{
        return(list(pedData=pedData,
                  idTrimmed=NA,
                  isTrimmed = FALSE,
                  bitSize = pedBits(pedData$id, pedData$father, pedData$mother)$bitSize))
      \}

    trimDat <- NULL
        
    for(idTrim in possiblyTrim)
      \{
        newPed <- removeSubj(removeId=idTrim, pedData)
        
        trimDat <- rbind(trimDat,
                         c(id=idTrim,
                           bitSize=pedBits(newPed$id, newPed$father, newPed$mother)$bitSize))
      \}

    bits <- trimDat[,2]

    # trim by subject with min bits. This trims fewer subject than
    # using max(bits).

    idTrim <- trimDat[bits==min(bits), 1]
          
    # break ties by random choice
    if(length(idTrim) > 1)
      \{
        rord <- order(runif(length(idTrim)))
        idTrim <- idTrim[rord][1]
      \}

    # overwrite old ped with new ped after trimming
    pedData <- removeSubj(removeId = idTrim, pedData)
    pedSize <- pedBits(pedData$id, pedData$father, pedData$mother)$bitSize
    
    return(list(pedData=pedData,
                idTrimmed = idTrim,
                isTrimmed = TRUE,
                bitSize = pedSize))
  \}


\nwendcode{}\nwbegindocs{118}\nwdocspar

\nwenddocs{}\nwbegincode{119}\moddef{pedigree.shrink.minor}\plusendmoddef

excludeUnavailFounders <- function(id, father, mother, avail)
  \{
    nOriginal <- length(id)
   
    zed <- father!=0 & mother !=0
    marriage <- paste(father[zed], mother[zed], sep="-" )
    sibship <- tapply(marriage, marriage, length)
    nm <- names(sibship)

    splitPos <- regexpr("-",nm)
    dad <- substring(nm, 1, splitPos-1)
    mom <- substring(nm, splitPos+1,  nchar(nm))
    
    ##  Want to look at parents with only one child.
    ##  Look for parents with > 1 marriage.  If any
    ##  marriage has > 1 child then skip this mom/dad pair.
    nmarr.dad <- table(dad)
    nmarr.mom <- table(mom)
    skip <- NULL
    if(any(nmarr.dad > 1)) \{
      ## Dads in >1 marriage
      ckdad <- which(as.logical(match(dad, names(nmarr.dad)[which(nmarr.dad > 1)],nomatch=FALSE)
))
      skip <- unique(c(skip, ckdad))
    \}
    if(any(nmarr.mom > 1)) \{
      ## Moms in >1 marriage
      ckmom <- which(as.logical(match(mom, names(nmarr.mom)[which(nmarr.mom > 1)],nomatch=FALSE)
))
      skip <- unique(c(skip, ckmom))
    \}
      
    if(length(skip) > 0) \{
      dad <- dad[-skip]
      mom <- mom[-skip]
      zed <- (sibship[-skip]==1) 
    \} else \{
      zed <- (sibship==1)
    \}
    
    n <- sum(zed)
    idTrimmed <- NULL
     if(n>0)
      \{
        # dad and mom are the parents of sibships of size 1
        dad <- dad[zed]
        mom <- mom[zed]
        for(i in 1:n)\{
          ## check if mom and dad are founders (where their parents = 0)
          dad.founder <- (father[id==dad[i]] == 0) & (mother[id==dad[i]] == 0)
          mom.founder <- (father[id==mom[i]] == 0) & (mother[id==mom[i]] == 0)
          both.founder <- dad.founder & mom.founder

          ## check if mom and dad have avail
          dad.avail <- avail[id==dad[i]]
          mom.avail <- avail[id==mom[i]]

          ## define not.avail = T if both mom & dad not avail
          not.avail <- (dad.avail==FALSE & mom.avail==FALSE)
        
          if(both.founder & not.avail)
            \{
              ## remove mom and dad from ped, and zero-out parent ids of their child
              child <- father==dad[i] & mother==mom[i]
              father[child] <- 0
              mother[child] <- 0

              idTrimmed <- c(idTrimmed, dad[i], mom[i])

              excludeParents <- (id!=dad[i]) & (id!=mom[i])
              id <- id[excludeParents]
              father <- father[excludeParents]
              mother <- mother[excludeParents]
              avail <- avail[excludeParents]
            \} 
        \}
      \}
    
    nFinal <- length(id)
    nTrimmed = nOriginal - nFinal

    return(list(nTrimmed = nTrimmed, idTrimmed=idTrimmed,
                id=id, father=father, mother=mother))
  \}


\nwendcode{}\nwbegindocs{120}\nwdocspar


\nwenddocs{}\nwbegincode{121}\moddef{pedigree.shrink.minor}\plusendmoddef
excludeStrayMarryin <- function(id, father, mother)\{
  # get rid of founders who are not parents (stray available marryins
  # who are isolated after trimming their unavailable offspring)

  trio <- data.frame(id=id, father=father, mother=mother)
  parent <- is.parent(id, father, mother)
  founder <- is.founder(father, mother)
  exclude <- !parent & founder
  trio <- trio[!exclude,,drop=FALSE]
  return(trio)
\}

trimAvailNonInform <- function(pedData)\{

  ## trim persons who are available but not informative b/c not parent
  ## by setting their availability to FALSE, then call pedTrim()
  checkParent <- is.parent(pedData$id, pedData$father, pedData$mother)

  for(i in 1:nrow(pedData))\{
    
    if(checkParent[i]==FALSE & pedData$avail[i]==TRUE & 
       all(as.matrix(pedData$affected)[i,]==0, na.rm=TRUE))
      \{
        fa <- pedData$father[i]
        mo <- pedData$mother[i]
        if(pedData$avail[pedData$id==fa]==TRUE & pedData$avail[pedData$id==mo]==TRUE)
          \{
            pedData$avail[i] <- FALSE
          \}
      \}
  \}
  
  datTrim <- pedTrim(pedData)
  pedData$father <- NULL  
  pedData$mother <- NULL
  pedData$avail <-  NULL
  pedData <- merge(datTrim, pedData)
  return(pedData)
\}

\nwendcode{}\nwbegindocs{122}\nwdocspar

Print a pedigree.shrink object.  Tell the original bit size and the trimmed bit size.

\nwenddocs{}\nwbegincode{123}\moddef{print.pedigree.shrink}\endmoddef
#$Log: print.pedigree.shrink.q,v $
#Revision 1.2  2009/11/19 14:35:01  sinnwell
#add ...
#
#Revision 1.1  2009/11/17 14:39:32  sinnwell
#Initial revision
#
#Revision 1.1  2008/07/16 20:23:14  sinnwell
#Initial revision
#

print.pedigree.shrink <- function(x, ...)\{

  printBanner(paste("Shrink of Pedigree ", unique(x$pedObj$ped), sep=""))
 
  cat("Pedigree Size:\\n")

  if(length(x$idTrimmed) > 2)
    \{
      n <- c(x$pedSizeOriginal, x$pedSizeIntermed, x$pedSizeFinal)
      b <- c(x$bitSize[1], x$bitSize[2], x$bitSize[length(x$bitSize)])
      row.nms <- c("Original","Only Informative","Trimmed")
    \} else \{
      n <- c(x$pedSizeOriginal, x$pedSizeIntermed)
      b <- c(x$bitSize[1], x$bitSize[2])
      row.nms <- c("Original","Trimmed")
    \}

  mat <- matrix(cbind(N = n, bitSize = b), ncol=2)
  dimnames(mat) <- list(row.nms, c("N.subj", "Bits"))
  print(mat, quote=FALSE)

  if(length(x$idTrimmed) > 2)
    \{
      cat("\\nSequential trimming of available informative subjects\\n")
      cat("(noninformative subjects not listed, NA = no trimming):\\n\\n")

      df <- data.frame(idTrimmed = x$idTrimmed[-1], bitsAfterTrim = x$bitSize[-1])
      print(df, quote=FALSE)
      cat("\\n")
    \} else \{
      cat("\\nNo Sequential trimming was necessary to achieve target bitSize\\n")
    \}
    
  ## cat("Pedigree after trimming:\\n")
  ## print(x$pedObj, quote=FALSE)
  invisible()
\}

\nwendcode{}\nwbegindocs{124}\nwdocspar

\nwenddocs{}\nwbegincode{125}\moddef{printBanner}\endmoddef
#$Log: printBanner.q,v $
#Revision 1.4  2007/01/23 21:00:27  sinnwell
#rm ending newline \\n.  Users can space if desired.
#
#Revision 1.3  2005/02/04 20:57:18  sinnwell
#banner.width now based on options()$width
#char.perline based on banner.width
#
#Revision 1.2  2004/06/25 15:56:48  sinnwell
#now compatible with R, changed end when a line is done
#
#Revision 1.1  2004/02/26 21:34:55  sinnwell
#Initial revision
#

printBanner <- function(str, banner.width=options()$width, char.perline=.75*banner.width, border = "=")\{

# char.perline was calculated taking the floor of banner.width/3

  vec <- str
  new<-NULL
  onespace<-FALSE
  for(i in 1:nchar(vec))\{
    if (substring(vec,i,i)==' ' && onespace==FALSE)\{
      onespace<-TRUE
      new<-paste(new,substring(vec,i,i),sep="")\}
    else if (substring(vec,i,i)==' ' && onespace==TRUE)
      \{onespace<-TRUE\}
    else\{
      onespace<-FALSE
      new<-paste(new,substring(vec,i,i),sep="")\}
  \}
  
  where.blank<-NULL
  indx <- 1
  
  for(i in 1:nchar(new))\{
    if((substring(new,i,i)==' '))\{
      where.blank[indx]<-i
      indx <- indx+1
    \}
  \}
  

# Determine the position in the where.blank vector to insert the Nth character position of "new"
  j<-length(where.blank)+1

# Add the Nth character position of the "new" string to the where.blank vector.
  where.blank[j]<-nchar(new)
  
  begin<-1
  end<-max(where.blank[where.blank<=char.perline])

# If end.ok equals NA then the char.perline is less than the position of the 1st blank.
  end.ok <- is.na(end) 

# Calculate a new char.perline. 
  if (end.ok==TRUE)\{ 
    char.perline <- floor(banner.width/2)
    end<-max(where.blank[where.blank<=char.perline])
  \}

  cat(paste(rep(border, banner.width), collapse = ""),"\\n")

  repeat \{
    titleline<-substring(new,begin,end)
    n <- nchar(titleline)
    if(n < banner.width)
      \{
        n.remain <- banner.width - n
        n.left <- floor(n.remain/2)
        n.right <- n.remain - n.left
        for(i in 1:n.left) titleline <- paste(" ",titleline,sep="")
        for(i in 1:n.right) titleline <- paste(titleline," ",sep="")
        n <- nchar(titleline)
      \}
     cat(titleline,"\\n")
    begin<-end+1
    end.old <- end
   # Next line has a problem when used in R.  Use print.banner.R until fixed.
   # Does max with an NA argument
    tmp <- where.blank[(end.old<where.blank) & (where.blank<=end.old+char.perline+1)]
    if(length(tmp)) end <- max(tmp)
    else break
   
#   end<-max(where.blank[(end.old<where.blank)&(where.blank<=end.old+char.perline+1)])
#   end.ok <- is.na(end)
#   if (end.ok==TRUE)
#      break
  \}
  
  cat(paste(rep(border, banner.width), collapse = ""), "\\n")
  invisible()
  
\}

\nwendcode{}\nwbegindocs{126}\nwdocspar


Plot a pedigree.shrink object, which calls the plot.pedigree function on the trimmed 
pedigree object.

\nwenddocs{}\nwbegincode{127}\moddef{plot.pedigree.shrink}\endmoddef
#$Log: plot.pedigree.shrink.q,v $
#Revision 1.4  2010/09/03 21:12:16  sinnwell
#use shrunk "avail" vector for the colored labels
#
#Revision 1.3  2009/11/19 14:57:18  sinnwell
#*** empty log message ***
#
#Revision 1.2  2009/11/17 23:09:51  sinnwell
#updated for ped object
#
#Revision 1.1  2008/07/16 20:23:38  sinnwell
#Initial revision
#

plot.pedigree.shrink <- function(x, bigped=FALSE, title="", ...)\{

  ##  Plot pedigrees, coloring available subjects according
  ##   to affection status.

  #col.subj <- ifelse(x$avail==TRUE & is.na(x$pedObj$affected==0), 3, 1)
  #col.subj <- ifelse(x$avail==TRUE & x$pedObj$affected==0, 4, col.subj)
  #col.subj <- ifelse(x$avail==TRUE & x$pedObj$affected==1, 2, col.subj)
  #col.subj <- ifelse(x$avail==TRUE & x$pedObj$affected==1, 5, col.subj)
  
  col.subj <- ifelse(x$avail==TRUE & is.na(x$pedObj$affected==0), 3, 1)
  col.subj <- ifelse(x$avail==TRUE & x$pedObj$affected==0, 4, col.subj)
  col.subj <- ifelse(x$avail==TRUE & x$pedObj$affected==1, 2, col.subj)
  col.subj <- ifelse(x$avail==FALSE & x$pedObj$affected==1, 5, col.subj)
  
  if(bigped==FALSE)\{
    tmp <- plot(x$pedObj, col=col.subj)
  \}

  if(bigped==TRUE)\{
    tmp <- plot.pedigree(x$pedObj, align=FALSE, packed=F,symbolsize=0.5,col=col.subj) #cex=0.25)
  \}
  browser()
  
  legend(c(max(tmp$x)*.75, max(tmp$y)*.9), 
         legend=c("Unavail+Unaff", "Avail+Aff","Avail+Unaff","Avail+Unk","UnAvail+Aff"),
         pch=c("*","*","*","*", "*"), col=c(1,2,4,3,5), lty=rep(1,5),bty="n")
  title(paste(title, "\\nbits = ", x$bitSize[length(x$bitSize)]))
\}


\nwendcode{}\nwbegindocs{128}\nwdocspar


/section{Pedigree Unrelated}

Purpose: Determine set of maximum number of unrelated
         available subjects from a pedigree
PI:      Dan Schaid
Author(s): Dan Schaid, Shannon McDonnell
Dates:   Created: 10/19/2007, Moved to mgenet: 11/3/2009

\nwenddocs{}\nwbegincode{129}\moddef{pedigree.unrelated}\endmoddef


#$Log: pedigree.unrelated.q,v $
#Revision 1.2  2010/02/11 22:36:48  sinnwell
#require kinship to be loaded before use
#
#Revision 1.1  2009/11/10 19:21:52  sinnwell
#Initial revision
#
#Revision 1.1  2009/11/03 16:42:27  sinnwell
#Initial revision
#
## Authors: Dan Schaid, Shannon McDonnell
## Updated by Jason Sinnwell

pedigree.unrelated <- function(ped, avail) \{
  
  # Requires: kinship function

  # Given vectors id, father, and mother for a pedigree structure,
  # and avail = vector of T/F or 1/0 for whether each subject
  # (corresponding to id vector) is available (e.g.,
  # has DNA available), determine set of maximum number
  # of unrelated available subjects from a pedigree.

  # This is a greedy algorithm that uses the kinship
  # matrix, sequentially removing rows/cols that
  # are non-zero for subjects that have the most
  # number of zero kinship coefficients (greedy
  # by choosing a row of kinship matrix that has
  # the most number of zeros, and then remove any
  # cols and their corresponding rows that are non-zero.
  # To account for ties of the count of zeros for rows,
  # a random choice is made. Hence, running this function
  # multiple times can return different sets of unrelated
  # subjects.

  id <- ped$id
  father <- ped$findex
  mother <- ped$mindex
  avail <- as.integer(avail)
  ##avail <- ped$status

  ## if kinship not loaded, tell to load it
  if(length(find("kinship"))==0) \{
    stop("library(kinship) needed\\n")
  \}
  
  kin <- kinship(id, father, mother)
  
  ord <- order(id)
  id <- id[ord]
  avail <- as.logical(avail[ord])
  kin <- kin[ord,][,ord]

  rord <- order(runif(nrow(kin)))

  id <- id[rord]
  avail <- avail[rord]
  kin <- kin[rord,][,rord]

  id.avail <- id[avail]
  kin.avail <- kin[avail,,drop=FALSE][,avail,drop=FALSE]

  diag(kin.avail) <- 0

  while(any(kin.avail > 0))
    \{
      nr <- nrow(kin.avail)
      indx <- 1:nrow(kin.avail)
      zero.count <- apply(kin.avail==0, 1, sum)
      
      mx <- max(zero.count[zero.count < nr])
      mx.zero <- indx[zero.count == mx][1]

      exclude <- indx[kin.avail[, mx.zero] > 0]

      kin.avail <- kin.avail[- exclude, , drop=FALSE][, -exclude, drop=FALSE]

    \}

  choice <- sort(dimnames(kin.avail)[[1]])
  
  return(choice)
\}


\nwendcode{}\nwbegindocs{130}\nwdocspar

\section{Checks}
Last are various helper routines and data checks.
\subsection{kindepth}
One helper function used throughout computes the depth of
each subject in the pedigree.  
For each subject this is defined as the maximal number of
generations of ancestors: how far to the farthest
founder.  
This can be called with a pedigree object, or with the 
full argument list.  In the former case we can simply
skip a step.
\nwenddocs{}\nwbegincode{131}\moddef{kindepth}\endmoddef
kindepth <- function(id, dad.id, mom.id, align=FALSE) \{
    if (class(id)=='pedigree' || class(id)=='pedigreeList') \{
        didx <- id$findex
        midx <- id$mindex
        n <- length(didx)
        \} 
    else \{
        n <- length(id)
        if (missing(dad.id) || length(dad.id) !=n)
            stop("Invalid father id")
        if (missing(mom.id) || length(mom.id) !=n)
            stop("Invalid mother id")
        midx <- match(mom.id, id, nomatch=0) # row number of my mom
        didx <- match(dad.id, id, nomatch=0) # row number of my dad
        \}
    if (n==1) return (0)  # special case of a single subject 
    parents <- which(midx==0 & didx==0)  #founders

    depth <- rep(0,n)
    # At each iteration below, all children of the current "parents" are
    #    labeled with depth 'i', and become the parents of the next iteration
    for (i in 1:n) \{
        child  <- match(midx, parents, nomatch=0) +
                  match(didx, parents, nomatch=0)

        if (all(child==0)) break
        if (i==n) 
            stop (paste("Impossible loop in the pedegree",
                      "(someone would have to be born after their own child)"))

        parents <- which(child>0) #next generation of parents
        depth[parents] <- i
        \}
    if (!align) return(depth)
\nwendcode{}\nwbegindocs{132}\nwdocspar

The align argument is used only by the plotting routines.  
It makes the plotted result prettier in the following (fairly common)
case. 
Assume that subjects A and B marry, we have some ancestry information for
both, and that A's ancestors go back 3 generations, B's for only two.
If we add +1 to the depth of B and all her ancestors, then A and B
will be the same depth, and will plot on the same line.
A marry-in to the pedigree with no ancestry is also handled nicely
by the algorithm.
However, if we have an inbred pedigree, there may not be a simple fix
of this sort.

The algorithm is
\begin{enumerate}
  \item Find any mother-father pairs that are mismatched in depth.
    We think that aligning the top of a pedigree is more important
    than aligning at the bottom, so choose a mismatch pair of minimal
    depth.
  \item The children's depth is max(father, mother) +1.  Call the
    parent closest to the children ``good'' and the other ``bad''.
  \item  Chase up the good side, and get a list of all subjects connected
    to "good", including in-laws (spouse connections) and sibs that are
    at this level or above.  Call this agood (ancestors of good).
    We do not follow any connections at a depth lower than the 
    marriage in question, to get the highest marriages right.
    For the bad side, just get ancestors.
  \item Avoid pedigree loops!  If the agood list contains anyone in abad,
    then don't try to fix the alignment, otherwise:
    Push abad down, then run the pushdown algorithm to
    repair any descendents --- you may have pulled down a grandparent but
    not the sibs of that grandparent.
\end{enumerate}
    
It may be possible to do better alignment when the pedigree has loops,
but it is definitely beyond this program's abilities.  This could be
an addition to authint one day.
One particular case that we've seen was a pair of brothers that married
a pair of sisters.  Pulling one brother down fixes the other at the
same time.
The code below, however, says "loop! stay away!".
\nwenddocs{}\nwbegincode{133}\moddef{kindepth}\plusendmoddef

    chaseup <- function(x, midx, didx) \{
        new <- c(midx[x], didx[x])  # mother and father
        new <- new[new>0]
        while (length(new) >1) \{
            x <- unique(c(x, new))
            new <- c(midx[new], didx[new])
            new <- new[new>0]
            \}
        x
        \}
        
    dads <- didx[midx>0 & didx>0]   # the father side of all spouse pairs
    moms <- midx[midx>0 & didx>0]
    # Get rid of duplicate pairs
    dups <- duplicated(dads + moms*n)
    if (any(dups)) \{
        dads <- dads[!dups]
        moms <- moms[!dups]
        \}
    npair<- length(dads)
    done <- rep(FALSE, npair)  #couples that are taken care of
    while (TRUE) \{
        pairs.to.fix <- (1:npair)[(depth[dads] != depth[moms]) & !done]
        if (length(pairs.to.fix) ==0) break
        temp <- pmax(depth[dads], depth[moms])[pairs.to.fix]
        who <- min(pairs.to.fix[temp==min(temp)])  # the chosen couple
        
        good <- moms[who]; bad <- dads[who]
        if (depth[dads[who]] > depth[moms[who]]) \{
            good <- dads[who]; bad <- moms[who]
            \}
        abad  <- chaseup(bad,  midx, didx)
        if (length(abad) ==1 && sum(c(dads,moms)==bad)==1) \{
            # simple case, a solitary marry-in
            depth[bad] <- depth[good]
            \}
        else \{
            agood <- chaseup(good, midx, didx)  #ancestors of the "good" side
            # For spouse chasing, I need to exclude the given pair
            tdad <- dads[-who]
            tmom <- moms[-who]
            while (1) \{
                # spouses of any on agood list
                spouse <- c(tmom[!is.na(match(tdad, agood))],
                            tdad[!is.na(match(tmom, agood))])
                temp <- unique(c(agood, spouse))
                temp <- unique(chaseup(temp, midx, didx)) #parents
                kids <- (!is.na(match(midx, temp)) | !is.na(match(didx, temp)))
                temp <- unique(c(temp, (1:n)[kids & depth <= depth[good]]))
                if (length(temp) == length(agood)) break
                else agood <- temp
                \}

            if (all(match(abad, agood, nomatch=0) ==0)) \{
                # shift it down
                depth[abad] <- depth[abad] + (depth[good] - depth[bad])
                #
                # Siblings may have had children: make sure all kids are
                #   below their parents.  It's easiest to run through the
                #   whole tree
                for (i in 0:n) \{
                    parents <- which(depth==i)
                    child <- match(midx, parents, nomatch=0) +
                             match(didx, parents, nomatch=0)
                    if (all(child==0)) break
                    depth[child>0] <- pmax(i+1, depth[child>0])
                    \}
                \}
            \}
        done[who] <- TRUE
        \}
    if (all(depth>0)) stop("You found a bug in kindepth's alignment code!")
    depth
    \}
\nwendcode{}\nwbegindocs{134}\nwdocspar

\subsection{familycheck}
\subsection{check.hint}
This routine tries to remove inconsistencies in spousal hints.
These and arise in autohint with complex pedigrees.
One can have ABA (subject A is on both the
left and the right of B), cycles, etc. 
Actually, these used to arise in autohint, I don't know if it's so
after the recent rewrite.
Users can introduce problems as well if they modify the hints.

\nwenddocs{}\nwbegincode{135}\moddef{check.hint}\endmoddef
check.hint <- function(hints, sex) \{
    if (is.null(hints$order)) stop("Missing order component")
    if (!is.numeric(hints$order)) stop("Invalid order component")
    n <- length(sex)
    if (length(hints$order) != n) stop("Wrong length for order component")
    
    spouse <- hints$spouse
    if (is.null(spouse)) hints
    else \{
        lspouse <- spouse[,1]
        rspouse <- spouse[,2]
        if (any(lspouse <1 | lspouse >n | rspouse <1 | rspouse > n))
            stop("Invalid spouse value")
        
        temp1 <- (sex[lspouse]== 'female' & sex[rspouse]=='male')
        temp2 <- (sex[rspouse]== 'female' & sex[lspouse]=='male')
        if (!all(temp1 | temp2))
            stop("A marriage is not male/female")
        
        hash <- n*pmax(lspouse, rspouse) + pmin(lspouse, rspouse)
        #Turn off this check for now - is set off if someone is married to two siblings
        #if (any(duplicated(hash))) stop("Duplicate marriage")

        # Break any loops: A left of B, B left of C, C left of A.
        #  Not yet done 
      \}
    hints
  \}
\nwendcode{}\nwbegindocs{136}\nwdocspar
\end{document}
\nwenddocs{}
